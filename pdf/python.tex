%% Generated by Sphinx.
\def\sphinxdocclass{jupyterBook}
\documentclass[a4paper,10pt,english,openany,oneside]{jupyterBook}
\ifdefined\pdfpxdimen
   \let\sphinxpxdimen\pdfpxdimen\else\newdimen\sphinxpxdimen
\fi \sphinxpxdimen=.75bp\relax
\ifdefined\pdfimageresolution
    \pdfimageresolution= \numexpr \dimexpr1in\relax/\sphinxpxdimen\relax
\fi
%% let collapsible pdf bookmarks panel have high depth per default
\PassOptionsToPackage{bookmarksdepth=5}{hyperref}
%% turn off hyperref patch of \index as sphinx.xdy xindy module takes care of
%% suitable \hyperpage mark-up, working around hyperref-xindy incompatibility
\PassOptionsToPackage{hyperindex=false}{hyperref}
%% memoir class requires extra handling
\makeatletter\@ifclassloaded{memoir}
{\ifdefined\memhyperindexfalse\memhyperindexfalse\fi}{}\makeatother

\PassOptionsToPackage{booktabs}{sphinx}
\PassOptionsToPackage{colorrows}{sphinx}

\PassOptionsToPackage{warn}{textcomp}

\catcode`^^^^00a0\active\protected\def^^^^00a0{\leavevmode\nobreak\ }
\usepackage{cmap}
\usepackage{fontspec}
\defaultfontfeatures[\rmfamily,\sffamily,\ttfamily]{}
\usepackage{amsmath,amssymb,amstext}
\usepackage{polyglossia}
\setmainlanguage{english}



\setmainfont{FreeSerif}[
  Extension      = .otf,
  UprightFont    = *,
  ItalicFont     = *Italic,
  BoldFont       = *Bold,
  BoldItalicFont = *BoldItalic
]
\setsansfont{FreeSans}[
  Extension      = .otf,
  UprightFont    = *,
  ItalicFont     = *Oblique,
  BoldFont       = *Bold,
  BoldItalicFont = *BoldOblique,
]
\setmonofont{FreeMono}[
  Extension      = .otf,
  UprightFont    = *,
  ItalicFont     = *Oblique,
  BoldFont       = *Bold,
  BoldItalicFont = *BoldOblique,
]



\usepackage[Bjarne]{fncychap}
\usepackage[,numfigreset=1,mathnumfig]{sphinx}

\fvset{fontsize=\small}
\usepackage{geometry}


% Include hyperref last.
\usepackage{hyperref}
% Fix anchor placement for figures with captions.
\usepackage{hypcap}% it must be loaded after hyperref.
% Set up styles of URL: it should be placed after hyperref.
\urlstyle{same}


\usepackage{sphinxmessages}


\usepackage{etoolbox}
\AtBeginEnvironment{figure}{\pretocmd{\hyperlink}{\protect}{}{}}  

        % Start of preamble defined in sphinx-jupyterbook-latex %
         \usepackage[Latin,Greek]{ucharclasses}
        \usepackage{unicode-math}
        % fixing title of the toc
        \addto\captionsenglish{\renewcommand{\contentsname}{Contents}}
        \hypersetup{
            pdfencoding=auto,
            psdextra
        }
        % End of preamble defined in sphinx-jupyterbook-latex %
        

\title{Video Holography}
\date{Jun 03, 2025}
\release{}
\author{Jan Genoe}
\newcommand{\sphinxlogo}{\sphinxincludegraphics{logo.png}\par}
\renewcommand{\releasename}{}
\makeindex
\begin{document}

\pagestyle{empty}
\sphinxmaketitle
\pagestyle{plain}
\sphinxtableofcontents
\pagestyle{normal}
\phantomsection\label{\detokenize{intro2::doc}}


\begin{DUlineblock}{0em}
\item[] \sphinxstylestrong{\Large Introduction}
\end{DUlineblock}

\sphinxAtStartPar
Today, despite many efforts by researchers world\sphinxhyphen{}wide, there are no holographic projectors that allow video\sphinxhyphen{}rate electronically controlled projection of complex holograms. Optically re\sphinxhyphen{}write\sphinxhyphen{}able holograms exist, but they are too slow; Acoustically\sphinxhyphen{}formed holograms can be switched fast but the image complexity is very limited. We identify the essential roadblock as one that we intend to clear by a breakthrough innovation coming from a combination of electronics, optics and material science.
We propose a radically novel way to make and control holograms, that will be based on the direct, analog, nanometer\sphinxhyphen{}resolution and nanosecond\sphinxhyphen{}speed control over the local refractive index of a slab waveguide core over several square centimetres. Holograms will be formed by leaky waves evanescent from the waveguide, and controlled by the refractive\sphinxhyphen{}index modulation profile in the core. That profile will be controlled and modulated by electrical fields applied with nano\sphinxhyphen{}precision through one of the cladding layers of the waveguide. To that end, a novel metamaterial is proposed for this cladding. Also novel driving schemes will be needed to control the new holographic projecting method.
With this combined radical innovation in architecture, materials and driving schemes, it is the goal of this project to fully prove the concept of video\sphinxhyphen{}rate electrically\sphinxhyphen{}controlled holographic projection. This will be the basis for many future innovations and applications, in domains such as augmented reality, automotive, optical metrology (LIDAR, microscopy, …), mobile communication, education, safety, etc…, and result in a high economic and social impact.

\begin{DUlineblock}{0em}
\item[] \sphinxstylestrong{\large Short history of Holography}
\end{DUlineblock}

\sphinxAtStartPar
The concepts of holography have been first elaborated by Dennis Gabor around 1947. He received the \sphinxhref{https://www.nobelprize.org/prizes/physics/1971/summary/}{Nobel Prize} for this work. However, at the time Dennis Gabor elaborated holography, neither (1) the coherent light sources nor (2) the technologies to pattern the hologram at sufficient high resolution were available. Good coherent laser sources in the visible came available around 1960, first for red and green and more recent also for the blue.

\sphinxAtStartPar
For static holograms, progress on the photographic film resolution was the first to move the resolution towards the quarter wavelength. Further improvement on the resolution of static holograms was obtained from the progress in photoresist resolution that was driven by the progress in the semiconductor industry. The advent of nano\sphinxhyphen{}imprint technologies further enabled upscaling of static holograms at large scale. Static holograms can be found in banknotes, credit cards, …

\sphinxAtStartPar
The next step is to make the hologram dynamic. This has been a challenging journey, as a hologram comprises a huge amount of information, which makes it a challenge to create, transfer and store this information. \hyperref[\detokenize{intro2:scalingroadmap}]{Fig.\@ \ref{\detokenize{intro2:scalingroadmap}}} shows the evolution of dynamic hologram demonstrators, both in resolution and speed.

\begin{figure}[htbp]
\centering
\capstart

\noindent\sphinxincludegraphics[width=1.000\linewidth]{{video-holography}.jpg}
\caption{Scaling Roadmap}\label{\detokenize{intro2:scalingroadmap}}\end{figure}

\begin{DUlineblock}{0em}
\item[] \sphinxstylestrong{\large Selected implementation in the Video holography ERC project}
\end{DUlineblock}

\sphinxAtStartPar
\hyperref[\detokenize{intro2:selectedimplementation}]{Fig.\@ \ref{\detokenize{intro2:selectedimplementation}}} shows the selected implementation that has been elaborated in this project. A 500 nm thick metamaterial separates the metal electrodes where the requested hologram is applied from the BTO waveguide. As a consequence, no metal is in the presence of the BTO waveguide, which allows light to propagate in the waveguide without scattering losses. The metamaterial is fabricated using SiN wherein vertical trenches are etched at 100 nm pitch in both directions. These trenches have been filled with InGaZnO that has been engineered such that the dielectric constant carefully matches the dielectric constant of SiN.  This lead to a metamaterial that is completely uniform and has no losses, when it is considered as the optical material that forms the cladding of the waveguide. However, when the same metamaterial is looked at from the electrical perspective, we have conductive channels at a 100 nm pitch that guides the electrical signal from the electrical contacts below to the waveguide above.
As the waveguide material, BTO has been selected. BTO is known to have the highest Pockels effect. This enables us to alter the effective dielectric constant of the waveguide locally at a pitch of 100 nm using relative small electric fields. This dielectric variation forms the hologram that is applied.

\begin{figure}[htbp]
\centering
\capstart

\noindent\sphinxincludegraphics[width=1.000\linewidth]{{video-holography2}.jpg}
\caption{Selected implementation}\label{\detokenize{intro2:selectedimplementation}}\end{figure}

\sphinxAtStartPar
Changing the hologram using the hardware in \hyperref[\detokenize{intro2:selectedimplementation}]{Fig.\@ \ref{\detokenize{intro2:selectedimplementation}}} can be obtained by changing the voltages at the bottom electrodes, which can be done reasonably fast, e.g. at 100 Hz rate. This allows also to swap the hologram between 3 subsequent holograms, one for red, one for green and one for blue at video rates. This yields full color video holography.

\begin{DUlineblock}{0em}
\item[] \sphinxstylestrong{\large Main project results}
\end{DUlineblock}

\sphinxAtStartPar
The project has focused on two major scientific challenges, i.e. the development of the metamaterial and the realization of a high\sphinxhyphen{}quality BTO waveguide.

\begin{DUlineblock}{0em}
\item[] \sphinxstylestrong{\large Metamaterial development}
\end{DUlineblock}

\sphinxAtStartPar
We have been able to fabricate the required metamaterial in a standard 300 mm cleanroom {[}\hyperlink{cite.bib:id9}{1}{]}. We have also modeled the obtained electrical fields in the BTO waveguides, both along the vertical axis and in the horizontal plane {[}\hyperlink{cite.bib:id8}{2}, \hyperlink{cite.bib:id6}{3}{]}. The knowledge of the Pockels coefficients both along the a\sphinxhyphen{}axis and the c\sphinxhyphen{}axis enables subsequently to describe a detailed algorithm for the hologram generation {[}\hyperlink{cite.bib:id9}{1}{]}.

\begin{DUlineblock}{0em}
\item[] \sphinxstylestrong{\large High\sphinxhyphen{}quality BTO waveguide}
\end{DUlineblock}

\sphinxAtStartPar
We have realized high\sphinxhyphen{}quality BTO layers {[}\hyperlink{cite.bib:id10}{4}{]} on Silicon wafers by both Molecular Beam Epitaxy (MBE) and Pulsed Laser Deposition (PLD) {[}\hyperlink{cite.bib:id13}{5}, \hyperlink{cite.bib:id14}{6}{]}. Both technologies required an SrTiO3 interface layer for lattice matching (see {[}\hyperlink{cite.bib:id12}{7}, \hyperlink{cite.bib:id11}{8}{]}).

\sphinxAtStartPar
The work on the BTO waveguides is been summarized in the PhD thesis of Tsang\sphinxhyphen{}Hsuan Wang {[}\hyperlink{cite.bib:id3}{9}{]}.

\begin{DUlineblock}{0em}
\item[] \sphinxstylestrong{\large Remaining challenges en further work}
\end{DUlineblock}

\sphinxAtStartPar
The control of the BTO waveguide at 100 nm resolution requires close interaction with the metamaterial. Our simulations (see {[}\hyperlink{cite.bib:id6}{3}{]}) indicate that when the separation between the BTO and the metamaterial goes beyond 5 nm, the effective control is too low for an efficient demonstrator. Therefor, we targeted an oxide\sphinxhyphen{}oxide bonding process yielding an separation below 2 nm. Although other demonstrators of oxide\sphinxhyphen{}oxide bonding, also in our lab, have indicated that this should be in reach, the practical between the BTO wafer and the metamaterial wafer has not yet been possible.

\begin{DUlineblock}{0em}
\item[] \sphinxstylestrong{\large Main funding info}
\end{DUlineblock}
\begin{itemize}
\item {} 
\sphinxAtStartPar
Programme Funding: Horizon 2020

\item {} 
\sphinxAtStartPar
Sub Programme Area: ERC\sphinxhyphen{}2016\sphinxhyphen{}ADG

\item {} 
\sphinxAtStartPar
Project Reference: 742299

\item {} 
\sphinxAtStartPar
From October 1, 2017 to March 31, 2023

\item {} 
\sphinxAtStartPar
Budget: EUR 2 499 074

\item {} 
\sphinxAtStartPar
Contract type: ERC\sphinxhyphen{}ADG

\item {} 
\sphinxAtStartPar
DOI: \sphinxhref{https://doi.org/10.3030/742299}{10.3030/742299}

\end{itemize}

\sphinxstepscope


\chapter{State\sphinxhyphen{}of\sphinxhyphen{}the\sphinxhyphen{}Art overview: Modulation mechanisms for dynamic holography}
\label{\detokenize{sota:state-of-the-art-overview-modulation-mechanisms-for-dynamic-holography}}\label{\detokenize{sota::doc}}
\begin{sphinxadmonition}{note}{Note:}
\sphinxAtStartPar
This chapter is reproduced with permission from Section 2.3.2 of the PhD of Guillaume Croes.
\end{sphinxadmonition}

\sphinxAtStartPar
In this section we compare the results obtained in the Video Holography ERC project with the state\sphinxhyphen{}of\sphinxhyphen{}the\sphinxhyphen{}art for dynamic holograms.
The results reported in the different subsections can be summarized in \hyperref[\detokenize{sota:sota}]{Fig.\@ \ref{\detokenize{sota:sota}}}.
The most relevant metrics are the hologram pixel resolution and the refresh rate.
The target for the hologram pixel resolution is defined by the 180 degree blue diffraction angle.
The target for the refresh rate is 360 Hz, as this allows to swap sufficiently fast the RGB colours of the 3 lasers without causing artifacts that are can be noticed.

\begin{sphinxuseclass}{cell}\begin{sphinxVerbatimOutput}

\begin{sphinxuseclass}{cell_output}
\begin{figure}[htbp]
\centering
\capstart

\noindent\sphinxincludegraphics{{9c6cf4d78b1941822ca8e9dd840e5a1ca09fd526327d0335836a32e8b5ae24c2}.png}
\caption{State\sphinxhyphen{}of\sphinxhyphen{}the\sphinxhyphen{}art overview of dynamic holographic projectors presented in terms of pixel resolution versus maximal hologram refresh rate.  The maximum diffraction angle that corresponds to the pixel pitch for each of the RGB colors is indicated on the right hand axis. We classify the reported results in terms of technology and hologram construction methodology. The domain allowing video\sphinxhyphen{}rate holography applications is enclosed by the dashed rectangle.  Only rewritable holography results have been included. When no actual refresh rate was reported, we included the result as Not Disclosed.}\label{\detokenize{sota:sota}}\end{figure}

\end{sphinxuseclass}\end{sphinxVerbatimOutput}

\end{sphinxuseclass}

\section{Doped Lithium Niobate}
\label{\detokenize{sota:doped-lithium-niobate}}
\sphinxAtStartPar
Various doped Lithium Niobate (LN) crystals show a photorefractive effect, i.e. when locally exposed to light, they change in refractive index.{[}\hyperlink{cite.bib:id331}{10}{]} Originally, this effect was thought to be optical damage to the LN, but Iron impurities were quickly found to be the root cause.{[}\hyperlink{cite.bib:id333}{11}, \hyperlink{cite.bib:id233}{12}{]} Afterwards, a wide range of dopants were employed, each offering different results in spectral response and writability.{[}\hyperlink{cite.bib:id232}{13}{]}

\sphinxAtStartPar
The effect originates due to redistribution of charges, which causes electric fields to form inside the crystal. Said fields, through the Pockels effect change the refractive index. Homogeneous illumination can be used to probe the modulated refractive index but has a tendency to erase the stored data depending on intensity and wavelength. Similarly, the stored data can be reset by a thermal anneal. More advanced crystals employ two separate dopants to create a photochromic behaviour in which the absorption changes when illuminated with a certain wavelength. One notable example is the combined use of Iron and Manganese, which form shallow and deep traps respectively.{[}\hyperlink{cite.bib:id311}{14}{]} Here, the shallow traps initially tend to be empty and can become populated after the deep traps are excited by ultraviolet (UV) light. The crystal then gains an absorption shoulder at longer wavelengths corresponding to the population of the shallow traps. This proofs useful for nonvolatile holographic storage as recordings can be made using a combination of short and long wavelengths, after which reading can be done using just the long wavelength for which the deep traps do not respond. This re\sphinxhyphen{}writability made this set of materials very interesting for holographic storage.{[}\hyperlink{cite.bib:id94}{15}{]} Unfortunately, the write\sphinxhyphen{}rewrite speed remains too low for any practical holographic display.


\section{Liquid Crystals}
\label{\detokenize{sota:liquid-crystals}}
\sphinxAtStartPar
Liquid crystals (LC) form a special phase of matter situated between crystalline and liquid phases. They form a collection of long molecules, often polymers, that can have a permanent electric dipole with a positive and negative charge on either side of the molecule. This enables them to be reoriented under influence of electric and magnetic fields. They are most known from their use in liquid crystal displays (LCD), which was one of the earliest flat panel displays. In the context of beam shaping and holography, they are mostly used in liquid crystal on silicon (LCOS) phase modulators. This is a rather mature platform, for which many commercial versions are available. That said, they are now also being considered as active component in metamaterials.

\sphinxAtStartPar
LCOS modulators were first introduced in the 1970’s by the Hughes aircraft company.{[}\hyperlink{cite.bib:id271}{16}{]} These initial devices employed an electric current instead of the now commonly used electric field, as driving force. Consequently, they weren’t suitable for phase modulation. That said, the prospect of a silicon backplane combined with the tunability of liquid crystals remained promising. Further research lead to more practical implementations capable of phase modulation around the early 1990’s.{[}\hyperlink{cite.bib:id177}{17}, \hyperlink{cite.bib:id393}{18}{]} From then onward, the technology gradually found its footing in industry but always stayed in the shadow of the much better known LCDs. LCOS was quickly used for beam steering applications and eventually for holography as well.{[}\hyperlink{cite.bib:id218}{19}, \hyperlink{cite.bib:id188}{20}{]} For a good overview on the types of LCOS, their applications and an industry perspective, the reader is referred to {[}\hyperlink{cite.bib:id270}{21}, \hyperlink{cite.bib:id191}{22}{]}. \hyperref[\detokenize{sota:sota}]{Fig.\@ \ref{\detokenize{sota:sota}}} contains several state of the art LCOS modulators that are currently commercially available.

\begin{figure}[htbp]
\centering
\capstart

\noindent\sphinxincludegraphics[width=1.000\linewidth]{{image2_11}.PNG}
\caption{a) Schematic of a tunable liquid crystal metamaterial. The metamaterial is patterned on the bottom electrode, whereas a brushed PVA layer ensures liquid crystal alignement at the top electrode. b) \sphinxhyphen{} c) Transmission behaviour for various wavelengths and bias of the LC cell shown in a). Supplying sufficient bias switches the transmission. The insets show the orientation of the liquid crystals. d) Photograph of holographic reconstruction projected by a LC spatial light modulator (SLM). e) Schematic of a LC cell employing electrodes to create tunable gratings. f) Outcoupling intensity for a grating with \(6\mu m\) period as shown in e). Images adapted from {[}\hyperlink{cite.bib:id188}{20}, \hyperlink{cite.bib:id165}{23}, \hyperlink{cite.bib:id143}{24}{]}.}\label{\detokenize{sota:chapter2-image11}}\end{figure}

\sphinxAtStartPar
More recently, liquid crystals are being considered as active component in tunable metasurfaces. Indeed, metasurfaces have shown excellent control over incident light in the context of lenses, plasmonics and beam shaping. On top of that, they have shown properties that are unachievable with normal materials such as perfect absorption and negative refractive indices.{[}\hyperlink{cite.bib:id308}{25}, \hyperlink{cite.bib:id178}{26}, \hyperlink{cite.bib:id49}{27}{]} However, their use cases remain limited due to their static nature. Hence, recent works aim to create tunable metasurfaces. Numerous, excellent reviews cover the field of dynamic metasurfaces.{[}\hyperlink{cite.bib:id136}{28}, \hyperlink{cite.bib:id268}{29}, \hyperlink{cite.bib:id449}{30}, \hyperlink{cite.bib:id451}{31}, \hyperlink{cite.bib:id396}{32}, \hyperlink{cite.bib:id454}{33}, \hyperlink{cite.bib:id30}{34}, \hyperlink{cite.bib:id56}{35}, \hyperlink{cite.bib:id386}{36}{]}. The groundwork for LC metasurfaces was laid when they were considered in split ring resonators (SRR) and core\sphinxhyphen{}shell nanosphere metamaterials.{[}\hyperlink{cite.bib:id305}{37}, \hyperlink{cite.bib:id159}{38}{]} Through the years, their tuning architecture and feature size was improved and downscaled respectively, pushing these devices gradually into the visible optical regime. However, rapid switching and individual pixel addressing remains absent. Multiple approaches attempt to tackle this challenge including SRR’s, embedded fishnet metamaterials, meta\sphinxhyphen{}atoms and LC enabled plasmonics.{[}\hyperlink{cite.bib:id165}{23}, \hyperlink{cite.bib:id143}{24}, \hyperlink{cite.bib:id23}{39}, \hyperlink{cite.bib:id83}{40}, \hyperlink{cite.bib:id157}{41}, \hyperlink{cite.bib:id112}{42}, \hyperlink{cite.bib:id450}{43}, \hyperlink{cite.bib:id121}{44}{]} A selected set of these is shown in \hyperref[\detokenize{sota:sota}]{Fig.\@ \ref{\detokenize{sota:sota}}} .


\section{Thermo\sphinxhyphen{}Optics}
\label{\detokenize{sota:thermo-optics}}
\sphinxAtStartPar
Thermo\sphinxhyphen{}optics employ heat to tune the optical properties of a material. This can be achieved through the thermo\sphinxhyphen{}optic coefficient i.e. the link between the refractive index and the temperature of a material but also through a phase change or transition. First phase change materials will be covered, after which devices using the thermo\sphinxhyphen{}optic coefficient will be highlighted.

\sphinxAtStartPar
Phase change materials form a collection of materials that have two or more crystalline phases with distinct optical properties. Through the years, they have become a staple material whenever a tunable meta\sphinxhyphen{}atom is desired. Commonly used materials include \(Ge_2Sb_2Te_5\) often abbreviated as GST, Vanadium Dioxide (\(VO_2\)), PbTe and liquid crystals. The desired phase change is thermally activated by raising the material’s temperature above the phase transitions temperature. For some materials such as \(VO_2\) and liquid crystals, the phase transition back to the original state occurs automatically. For others such as GST, an amorphous highly resistive phase is typically encountered after deposition. Heating up to about \(150^{\circ}\)C leads to crystallization into a stable more metallic phase having lower resistivity. To return to the initial amorphous state, GST needs to be heated above is melting point (\(600^{\circ}\)C) and quickly cooled such that it cannot form a metallic lattice. Heating can either be done optically, by a focused laser or through the inclusion of heating elements enabling current generated Joule heating.

\begin{figure}[htbp]
\centering
\capstart

\noindent\sphinxincludegraphics[width=1.000\linewidth]{{image2_12_2}.PNG}
\caption{a)Optical behaviour of commonly employed phase change material GST, which can be switched between amorphous and crystalline phases. b) The effect of the different phases of a thin GST layer have on the reflectivity of an optical stack. c) Holographic projected of a GST patterned metasurface made tunable by laser scribing. d) Artists impression of the operation of a Magnesium based metasurface, which can undergo a phase change due to hydrogenation of the meta\sphinxhyphen{}atoms. e) SEM image of Magnesium meta\sphinxhyphen{}atoms and their scattering behaviour versus time. f) Holographic projection fo the device shown in d) and e). The optical is shown for various states of the phase change material to showcase its tunability. g) Artists impression of a GST metasurface, showing both write and read lasers. h) Optical behaviour of a GST metalens, ecoded in to the laser by laser writing. Images adapted from {[}\hyperlink{cite.bib:id227}{45}, \hyperlink{cite.bib:id17}{46}, \hyperlink{cite.bib:id332}{47}{]}.}\label{\detokenize{sota:chapter2-image12}}\end{figure}

\sphinxAtStartPar
Metasurfaces employing phase change materials have shown excellent results in a multitude of applications ranging from transmission and reflection tuning {[}\hyperlink{cite.bib:id417}{48}, \hyperlink{cite.bib:id372}{49}, \hyperlink{cite.bib:id374}{50}, \hyperlink{cite.bib:id62}{51}, \hyperlink{cite.bib:id474}{52}{]} to beam steering {[}\hyperlink{cite.bib:id173}{53}, \hyperlink{cite.bib:id155}{54}{]}.

\sphinxAtStartPar
For example, modulation depths of up to \(90\%\) have been achieved for absorption tuning {[}\hyperlink{cite.bib:id148}{55}{]}, relative transmission changes of \(500\%\) have been reported {[}\hyperlink{cite.bib:id76}{56}{]} and beam steering angles up to \(40^{\circ}\) have been shown.{[}\hyperlink{cite.bib:id69}{57}{]}

\sphinxAtStartPar
Additionally, slightly more advanced metasurfaces have been used to create tunable metalenses.{[}\hyperlink{cite.bib:id332}{47}, \hyperlink{cite.bib:id82}{58}{]}

\sphinxAtStartPar
Several approaches have even been able to go even one step further and have shown beam shaping and holography. Indeed, holographic metasurfaces have been made by hydrogenation\sphinxhyphen{}dehydrogenation of Mg meta\sphinxhyphen{}atoms {[}\hyperlink{cite.bib:id17}{46}{]}, tuning of a resonance {[}\hyperlink{cite.bib:id227}{45}, \hyperlink{cite.bib:id332}{47}, \hyperlink{cite.bib:id28}{59}, \hyperlink{cite.bib:id387}{60}, \hyperlink{cite.bib:id44}{61}{]}, and tunable split ring resonators.{[}\hyperlink{cite.bib:id437}{62}{]} Pixel sizes down to \(600nm\) have been achieved for devices operating in On\sphinxhyphen{}Off state (\(50\)s switch time). On the other hand, faster switching (\(500ns\) rise time \sphinxhyphen{} \(100\mu s\) fall time) is possible at slightly larger pixel size (\(4\mu m\)).

\sphinxAtStartPar
It should be noted that these approaches mostly employ longer wavelengths starting from the near IR to THz frequencies. On top of that, the meta\sphinxhyphen{}atom switching rate remains limited at the moment given that integrated heaters are only included in a minority of works such that others either rely on a laser pulse or hot plate for write\sphinxhyphen{}rewrite. The most impressive result currently has been achieved by SWAVE Photonics which managed to incorporate a modulated phase change material into a complete display stack, reaching pixel sizes down to \(300\)nm and video\sphinxhyphen{}rate capable refresh rates. This display almost reaches the requirements for a true videoholographic display.{[}\hyperlink{cite.bib:id412}{63}{]}

\begin{figure}[htbp]
\centering
\capstart

\noindent\sphinxincludegraphics[width=1.000\linewidth]{{image2_12}.PNG}
\caption{a)\sphinxhyphen{} b) Schematic and microscope image of a waveguide fed beam steerer that employs thermo\sphinxhyphen{}optics phase shifters and outcoupling gratings. c) Schematic of an advanced \(8\) by \(8\) optical phased array that has independent control over each outcoupling element. d) Simulation and measured results from c), by applying various bias conditions. e) Schematic of a 2D beam steerer employing a lattice\sphinxhyphen{}shifted photonic crystal waveguide in combination with a prism lens. f) Measured farfield outcoupling from the device in patterns from the device in e). Images adapted from {[}\hyperlink{cite.bib:id330}{64}, \hyperlink{cite.bib:id258}{65}, \hyperlink{cite.bib:id496}{66}{]}.}\label{\detokenize{sota:chapter2-image13}}\end{figure}

\sphinxAtStartPar
Alternatively, the thermo\sphinxhyphen{}optic coefficient can be used to tune the refractive index more directly. Here, at low temperatures, any variation leads to a linear change in refractive index. The effect is typically very small with coefficients ranging from \(10^{-6}\) to \(10^{-3} /^{\circ}C\).{[}\hyperlink{cite.bib:id438}{67}{]} Nevertheless, the effect is often used in waveguide or resonator structures given that it enables extremely fine control or that the resonance leads to amplification of the effect. Thermo\sphinxhyphen{}optically tuned waveguide modulators form an active topic in the field of light detection and ranging (LIDAR), since they can be used to create on chip beam steering platforms. Both one\sphinxhyphen{}dimensional (1D) {[}\hyperlink{cite.bib:id330}{64}, \hyperlink{cite.bib:id241}{68}, \hyperlink{cite.bib:id131}{69}{]} and 2D beam steering based on thermo\sphinxhyphen{}optical modulation has been shown.{[}\hyperlink{cite.bib:id258}{65}, \hyperlink{cite.bib:id496}{66}, \hyperlink{cite.bib:id466}{70}, \hyperlink{cite.bib:id463}{71}, \hyperlink{cite.bib:id439}{72}, \hyperlink{cite.bib:id263}{73}, \hyperlink{cite.bib:id264}{74}{]} One noteworthy example was the creation of a \(64\) by \(64\) optical phased array with individual phase shifters for each emitter, which theoretically could be used to create a holographic display.


\section{Acousto\sphinxhyphen{}Optics}
\label{\detokenize{sota:acousto-optics}}
\sphinxAtStartPar
Acousto\sphinxhyphen{}optics offer a straightforward method to dynamically control incident light. Standard acousto\sphinxhyphen{}optic modulators only require three components, namely a grating or prism coupler, an acoustic\sphinxhyphen{}optic waveguide and an ultrasound transducer. Here, the grating or prism ensures efficient coupling to a guided mode and the transducer creates a surface acoustic wave. In doing so, the refractive index of the waveguide can be tuned dynamically by the compression and expansion linked to the travelling acoustic wave.  The effect is a specific type of photo\sphinxhyphen{}elasticity for which is known that mechanical strain leads to changes in permittivity. Similar to the electro\sphinxhyphen{}optical effect, a tensor calculation is required to understand the full scope of the modulation. When the applied surface acoustic wave meets the Bragg condition, light can be coupled out of the waveguide. Nowadays, LN is commonly used as acousto\sphinxhyphen{}optic material given that it has a strong response combined with low optical waveguide losses.

\begin{figure}[htbp]
\centering
\capstart

\noindent\sphinxincludegraphics[width=1.000\linewidth]{{image2_13}.PNG}
\caption{a)Schematics of the side and top view of a waveguide based acousto\sphinxhyphen{}optic spatial light modulator. b) Schematic of a free space acousto\sphinxhyphen{}optic holographic projector. c) Measured holographic projections at various detector integration time employing the device from b). d) Observed holographic projection employing a device structure similar to a). Multicolors were achieved through wavelength superpositioning. Images adapted from {[}\hyperlink{cite.bib:id262}{75}, \hyperlink{cite.bib:id98}{76}, \hyperlink{cite.bib:id18}{77}{]}.}\label{\detokenize{sota:chapter2-image14}}\end{figure}

\sphinxAtStartPar
Acousto\sphinxhyphen{}optic beam deflectors on LN have been around since the \(1970\)’s.{[}\hyperlink{cite.bib:id210}{78}{]} Initially, these devices contained a single modulated channel capable of \(1\)D beam steering.{[}\hyperlink{cite.bib:id67}{79}, \hyperlink{cite.bib:id89}{80}{]} Devices were limited in frequency and thus could only supply acoustic waves capable of creating gratings coupling between guided and cladding or substrate modes. As such, steered light exited the device at the end of the LN wafer. The addition of a second transducer eventually lead to 2D beam steering.{[}\hyperlink{cite.bib:id211}{81}{]} In more recent years, a larger degree of control was attained by using the device in a leaky mode state instead.{[}\hyperlink{cite.bib:id399}{82}{]} Here, by modulating the device at higher frequencies, guided mode to radiation mode coupling is enabled such that light can be coupled out along its entire surface. This concept has been applied to both beam steering for LIDAR and visible holography.{[}\hyperlink{cite.bib:id262}{75}, \hyperlink{cite.bib:id98}{76}, \hyperlink{cite.bib:id18}{77}, \hyperlink{cite.bib:id184}{83}{]} More specifically, a holographic display reaching about \(5\)Hz in refresh rate and a pixel size of \(12\mu m\), calculated from the applied frequency, was achieved. The device excelled in its resolution as it was capable of creating compositional images having up to \(355200\) pixels by \(156\) pixels. Consequently, it is one of the best implementations of holographic display technology yet.


\section{Carrier Injection}
\label{\detokenize{sota:carrier-injection}}
\sphinxAtStartPar
TCOs have emerged since the beginning of 21st century as a crucial component in solar cells and flat panel displays.{[}\hyperlink{cite.bib:id444}{84}, \hyperlink{cite.bib:id446}{85}, \hyperlink{cite.bib:id448}{86}{]} They excel, for example, as transparent electrodes or thin film transistors. This is due to their unique optical and electrical properties that combine good conductivity with low absorption. TCO conductivity can vary widely between values typically attributed to dielectrics and semiconductors depending on the amount of present carriers. Stoichiometric TCOs are in general more dielectric. Conversely, larger conductivities comparable to semiconductors, require more free carriers to be present which for most TCOs can be solved by creating oxygen vacancies.{[}\hyperlink{cite.bib:id339}{87}, \hyperlink{cite.bib:id341}{88}{]} Tuning of TCO properties is easily achieved through deposition parameters and post deposition anneals.{[}\hyperlink{cite.bib:id338}{89}, \hyperlink{cite.bib:id498}{90}, \hyperlink{cite.bib:id334}{91}{]} Next to that, a wide variety of TCOs such as Indium Tin Oxide (ITO), Zinc Oxide (ZnO), Indium Gallium Zinc Oxide (IGZO), … have been investigated. Interestingly, due to their oxide behaviour they tend to have remarkably low absorption accompanying their electrical behaviour. TCOs thus occupy a rather rare position in the semiconductor realm and are now considered as backbone for the next generation of plasmonics and in epsilon near zero and near zero index materials.{[}\hyperlink{cite.bib:id86}{92}, \hyperlink{cite.bib:id445}{93}, \hyperlink{cite.bib:id447}{94}{]}

\sphinxAtStartPar
Their uniqueness however, does not end there. About a decade ago, it was found that the permittivity of TCOs can be actively modulated through the injection or extraction of carriers.{[}\hyperlink{cite.bib:id472}{95}{]} By employing indium tin oxide (ITO) as active layer in a a metal\sphinxhyphen{}oxide\sphinxhyphen{}semiconductor heterostructure, a thin charge accumulation layer could be formed at the interface. Here, the carrier density could be altered between \(10^{18}\)cm\(^{-1}\) and \(10^{23}\)cm\(^{-1}\) resulting in a \(5nm\) layer in which a refractive index change (\(\Delta n\)) of \(1.39\) was recorded at \(800nm\). Optically, this behaviour can be described by a Drude model which links the carrier concentration to the permittivity. The modulation is primarily prevalent in the infrared as free electrons influence optical properties here, but a tail of the effect stretches up to the visible regime.  Even though the effect only occurs at the interface, its exceptional size makes it a viable candidate for optical modulators, ideally through the use of ultra\sphinxhyphen{}thin layers (\(<10nm\)) to limit optical losses and maximize modulation. Lastly, it should be mentioned that the modulation is fast when compared to other techniques since it is only limited by its \(RC\) time constant.

\begin{figure}[htbp]
\centering
\capstart

\noindent\sphinxincludegraphics[width=1.000\linewidth]{{image2_14}.PNG}
\caption{a) Variation of refractive index of transparent conductive oxides through carrier accumulation. Measured by ellipsometry in a thin accumulation layer. b) Schematic of a grating structure, covered with ITO. Through accumulation of charges the behaviour of the grating can be tuned. c) Observed reflectivity change from the device in b). The inset shows the percieved change in permittivity. d) Artists impression of a plasmonic metasurface grating, tuned though carrier injection. Incident light is steerd by the applied bias. A single nano\sphinxhyphen{}resonator is highlighted, indicating where the accumulation layer resides. e) \sphinxhyphen{} f) Reflectivity and phase change under positive and negative bias, observed for the device shown in d). Images adapted from {[}\hyperlink{cite.bib:id472}{95}, \hyperlink{cite.bib:id71}{96}, \hyperlink{cite.bib:id84}{97}{]}.}\label{\detokenize{sota:chapter2-image15}}\end{figure}

\sphinxAtStartPar
To date, the effect has been applied in tunable epsilon near zero materials {[}\hyperlink{cite.bib:id468}{98}, \hyperlink{cite.bib:id158}{99}{]}, plasmonic modulators {[}\hyperlink{cite.bib:id349}{100}, \hyperlink{cite.bib:id354}{101}, \hyperlink{cite.bib:id162}{102}, \hyperlink{cite.bib:id304}{103}{]} and a variety of beam steering applications. This last topic was pioneered by a gate tunable metasurface constructed from a Gold \sphinxhyphen{} ITO \sphinxhyphen{} Aluminium Oxide back plane on which a Gold grating electrode was patterned to enable MIM plasmonic modulation. Here, the grating serves as reflection antenna which can be modulated by applying electrical bias to both gold electrodes, in doing so changing reflection characteristics.{[}\hyperlink{cite.bib:id193}{104}{]} At an incident wavelength of \(1550nm\) and \(2.5V\) bias a normalized reflectance change of \(28.9\%\) and phase shift of \(180^{\circ}C\) was found. Beam steering was enabled by biasing periodically with varying voltage, which allowed switching between \(0\) order and \(-1\) and \(+1\) order reflection. Changing the periodicity of the applied bias tunes the steering angle. Afterwards, both amplitude and phase modulation metasurfaces implementing TCOs were investigated. Amplitude modulation proved especially interesting in tunable absorbers which often utilize a similar MIM structure that acts as a tunable resonant cavity showing a reflectance change of up to \(82\%\) at \(1550nm\).{[}\hyperlink{cite.bib:id71}{96}, \hyperlink{cite.bib:id158}{99}, \hyperlink{cite.bib:id197}{105}, \hyperlink{cite.bib:id196}{106}{]} On the other hand, TCO based phase modulators have steadily been improved towards full \(2\pi\) phase modulation.{[}\hyperlink{cite.bib:id147}{107}{]} Currently, phase modulation up to \(300^{\circ}\)C has been shown in the infrared.{[}\hyperlink{cite.bib:id145}{108}{]} Next to that, phase modulation devices using carrier injection have shown beam steering, LIDAR and beam focusing.{[}\hyperlink{cite.bib:id84}{97}, \hyperlink{cite.bib:id465}{109}, \hyperlink{cite.bib:id166}{110}{]}
To my knowledge, no TCOs based modulators have been implemented into a holographic display even though this could be achieved by a 2D array of individually addressed elements.


\section{Micro\sphinxhyphen{}electromechanical Systems}
\label{\detokenize{sota:micro-electromechanical-systems}}
\sphinxAtStartPar
Micro\sphinxhyphen{}electromechanical systems (MEMS) are a well established technology that form a bridge between typical silicon based electronic driving and mechanical movement. In doing so, MEMS create a unique set of capabilities that proved relevant in sensors (inertial and pressure), optical scanning and surface probes. Commonly used device actuation schemes are based on electrostatics, thermoelectric, piezoelectrics and electromagnetic effects. Of these, electrostatics and thermoelectrics are most used. Electrostatic based MEMS offer a fast response, lower power consumption and ease of fabrication.{[}\hyperlink{cite.bib:id275}{111}{]} Thermoelectric MEMS, on the other hand, provide slower modulation and higher power consumption but are often used in out\sphinxhyphen{}of\sphinxhyphen{}plane actuation. Manufacturing\sphinxhyphen{}wise these MEMS types are compatible with complementary metal oxide semiconductor (CMOS) technologies as they leverage many of the same principles. Both piezoelectric and electromagnetic approaches require more uncommon materials, and thus are not as prominently used.{[}\hyperlink{cite.bib:id283}{112}, \hyperlink{cite.bib:id371}{113}{]}

\begin{figure}[htbp]
\centering
\capstart

\noindent\sphinxincludegraphics[width=1.000\linewidth]{{image2_15}.PNG}
\caption{a) Artists impression of a split ring resonator actuated by electrostatic MEMS. b) SEM image of a single unit cell of the device in a). c) Schematic o a doublet metalens built on a MEMS actuator to provide tunability. d)Microscope and SEM images of the doublet metalens constituents shown in c) e) Simulated and measured holographic projection from a grating metasurface actuated by MEMS. f) SEM image of MEMS actuated gratings used in e). Images adapted from {[}\hyperlink{cite.bib:id68}{114}, \hyperlink{cite.bib:id277}{115}, \hyperlink{cite.bib:id37}{116}{]}.}\label{\detokenize{sota:chapter2-image16}}\end{figure}

\sphinxAtStartPar
Due to their unique tuning capabilities MEMS are now also considered as tunable element in metamaterials. Here they can serve two roles, either they add tunability to a metasurface as a whole or they provide tunability to each individual meta\sphinxhyphen{}atom. The first scenario has for example, been applied to metalenses which by positioning them on a MEMS actuator can be used for dynamically steering the focusing point.{[}\hyperlink{cite.bib:id277}{115}, \hyperlink{cite.bib:id368}{117}, \hyperlink{cite.bib:id146}{118}, \hyperlink{cite.bib:id279}{119}{]} Other approaches have achieved beam steering at visible frequencies by tuning a cavity grating and transmission tuning of up to \(80\%\).{[}\hyperlink{cite.bib:id68}{114}, \hyperlink{cite.bib:id418}{120}, \hyperlink{cite.bib:id289}{121}{]}
These devices are relatively easy to manufacture and might prove useful in sensing in LIDAR applications. They do however not offer complete reprogrammable phase profiles. Indeed, more advanced tunability requires individually addressed pixels, which quickly drives up device complexity. Device arrays up to \(160\) by \(160\) pixels have been reported and have achieved beam steering at THz and infrared frequencies.{[}\hyperlink{cite.bib:id37}{116}, \hyperlink{cite.bib:id39}{122}{]}

\sphinxAtStartPar
Limited efforts have attempted to create a MEMS driven holographic display.
That said, MEMS holographic projection was achieved by creating phased arrays and metal insulator metal cavities. Phase differences between individual pixels were created by lateral displacement of reflection gratings (Lohmann) and by cantilever based tuning of a plasmonic resonance respectively.{[}\hyperlink{cite.bib:id37}{116}, \hyperlink{cite.bib:id75}{123}, \hyperlink{cite.bib:id74}{124}{]}
In general, these attempts have been hindered by the pixel size of MEMS which currently still around the micrometer to tens of micrometer range. As such, the examples mentioned above also operate at infrared wavelengths to retain adequate control.  More advanced beam steering, shaping and holography require modulation at a subwavelength scale. Further downscaling of MEMS leads to, so called, nano\sphinxhyphen{}electromechanical systems.{[}\hyperlink{cite.bib:id303}{125}{]} These devices do attain the desired modulator scale, but again bring about complex design.

\sphinxstepscope


\chapter{ERC Team}
\label{\detokenize{Team:erc-team}}\label{\detokenize{Team::doc}}
\sphinxAtStartPar
The ERC project Video Holography has been executed at \sphinxhref{https://www.imec.be}{imec} as the host institute. It has been conceived and xx by Prof. Jan Genoe, as the principal investigator. He has been supported a strong team of senior academic staff (see \hyperref[\detokenize{Team:staff}]{Table \ref{\detokenize{Team:staff}}}) and two PhD students diving deep into the subject of the project (see \hyperref[\detokenize{Team:phdstaff}]{Table \ref{\detokenize{Team:phdstaff}}}). The project would also not have been possible without the strong support from several other technology experts from the different research units in the host institute \sphinxhref{https://www.imec.be}{imec}.


\section{Core Team}
\label{\detokenize{Team:core-team}}

\subsection{Principal Investigator}
\label{\detokenize{Team:principal-investigator}}

\begin{savenotes}\sphinxattablestart
\sphinxthistablewithglobalstyle
\centering
\begin{tabular}[t]{\X{25}{125}\X{100}{125}}
\sphinxtoprule
\sphinxtableatstartofbodyhook
\sphinxAtStartPar
\sphinxincludegraphics{{JanGenoe}.jpg}
&
\sphinxAtStartPar
Prof. Jan Genoe is scientific director at the Host institution imec and has received all support from the Host institution to build the research team and execute the research. Prof. Jan Genoe also takes the scientific leadership of the Video Holography ERC research.
\\
\sphinxbottomrule
\end{tabular}
\sphinxtableafterendhook\par
\sphinxattableend\end{savenotes}


\subsection{Senior academic staff in the team}
\label{\detokenize{Team:senior-academic-staff-in-the-team}}

\begin{savenotes}\sphinxattablestart
\sphinxthistablewithglobalstyle
\centering
\sphinxcapstartof{table}
\sphinxthecaptionisattop
\sphinxcaption{Senior academic staff}\label{\detokenize{Team:staff}}
\sphinxaftertopcaption
\begin{tabular}[t]{\X{25}{125}\X{100}{125}}
\sphinxtoprule
\sphinxtableatstartofbodyhook
\sphinxAtStartPar
\sphinxincludegraphics{{Robert}.jpg}
&
\sphinxAtStartPar
Dr. Robert Gehlhaar provides scientific input on the optical stack design and characterization.
\\
\sphinxhline
\sphinxAtStartPar
\sphinxincludegraphics{{Zsolt}.jpg}
&
\sphinxAtStartPar
Dr. Zsolt Tokei provides technology input on the realisation of devices in the 300mm cleanroom.
\\
\sphinxhline
\sphinxAtStartPar
\sphinxincludegraphics{{Clement}.jpg}
&
\sphinxAtStartPar
Prof. Clement Merckling provides scientific input on the material growth conditions for the BTO and STO waveguide materials.
\\
\sphinxhline
\sphinxAtStartPar
\sphinxincludegraphics{{Paul}.jpg}
&
\sphinxAtStartPar
Prof. Paul Heremans provides scientific input on the device performance.
\\
\sphinxbottomrule
\end{tabular}
\sphinxtableafterendhook\par
\sphinxattableend\end{savenotes}


\subsection{PhD students}
\label{\detokenize{Team:phd-students}}

\begin{savenotes}\sphinxattablestart
\sphinxthistablewithglobalstyle
\centering
\sphinxcapstartof{table}
\sphinxthecaptionisattop
\sphinxcaption{PhD students}\label{\detokenize{Team:phdstaff}}
\sphinxaftertopcaption
\begin{tabular}[t]{\X{25}{125}\X{100}{125}}
\sphinxtoprule
\sphinxtableatstartofbodyhook
\sphinxAtStartPar
\sphinxincludegraphics{{Guillaume}.jpg}
&
\sphinxAtStartPar
Guillaume Croes is the PhD student elaborating the metamaterial stack and optical model for the optimization for driving the hologram.
He was awarded a Strategic Basic Research Fellowship from FWO for his PhD
\\
\sphinxhline
\sphinxAtStartPar
\sphinxincludegraphics{{Tsang-HsuanWang}.jpg}
&
\sphinxAtStartPar
Tsang\sphinxhyphen{}Hsuan Wang is the PhD student elaborating the optimized growth conditions for the BTO and STO waveguide materials.
\\
\sphinxbottomrule
\end{tabular}
\sphinxtableafterendhook\par
\sphinxattableend\end{savenotes}


\section{Other contributors}
\label{\detokenize{Team:other-contributors}}\begin{itemize}
\item {} 
\sphinxAtStartPar
Diana Tsvetanova provides input on the CMP processes in the 300 mm line.

\item {} 
\sphinxAtStartPar
Yunlong Li provides input on the process sequence in the 300 mm line.

\item {} 
\sphinxAtStartPar
Renauld Puybaret is in charge of the daily supervision of the process in the 300 mm line.

\item {} 
\sphinxAtStartPar
Thomas Raes is in charge of the Mask preparation for the process in the 300 mm line.

\item {} 
\sphinxAtStartPar
Deniz Sabuncuoglu Tezcan is in charge of the supervision of the process in the 300 mm line.

\item {} 
\sphinxAtStartPar
Jeremy Segers is in charge of the oxide\sphinxhyphen{}oxide bonding process between the BTO wafer and the optical transparent metamaterial.

\end{itemize}

\sphinxstepscope


\chapter{ERC Publications}
\label{\detokenize{Publications2:erc-publications}}\label{\detokenize{Publications2::doc}}
\sphinxAtStartPar
The work performed in the ERC project Video Holography has been published in {\hyperref[\detokenize{Publications2:journal-target}]{\sphinxcrossref{\DUrole{std,std-ref}{journal papers}}}} and presented at {\hyperref[\detokenize{Publications2:conferences-target}]{\sphinxcrossref{\DUrole{std,std-ref}{conferences}}}}. A more elaborated description of the results can be found in the {\hyperref[\detokenize{Publications2:thesis-target}]{\sphinxcrossref{PhDs}}} that have been supported by this ERC.


\section{Journal papers}
\label{\detokenize{Publications2:journal-papers}}\label{\detokenize{Publications2:journal-target}}\begin{itemize}
\item {} 
\sphinxAtStartPar
Tsang\sphinxhyphen{}Hsuan Wang,
Po\sphinxhyphen{}Chun Hsu,
Maxim Korytov,
Jan Genoe,
Clement Merckling,
\sphinxstylestrong{Polarization control of epitaxial barium titanate (BaTiO3) grown by pulsed\sphinxhyphen{}laser deposition on a MBE\sphinxhyphen{}SrTiO3/Si(001) pseudo\sphinxhyphen{}substrate},
Journal of Applied Physics 128, 104104 (September 2020),
\sphinxhref{http://dx.doi.org/10.1063/5.0019980}{DOI: 10.1063/5.0019980}

\item {} 
\sphinxAtStartPar
Tsang\sphinxhyphen{}Hsuan Wang,
Robert Gehlhaar,
Thierry Conard,
Paola Favia,
Jan Genoe,
Clement Merckling,
\sphinxstylestrong{Interfacial control of SrTiO3/Si(001) epitaxy and its effect on physical and optical properties},
Journal of Crystal Growth 582, 126524 (March 2022),
\sphinxhref{http://dx.doi.org/10.1016/j.jcrysgro.2022.126524}{DOI: 10.1016/j.jcrysgro.2022.126524}

\item {} 
\sphinxAtStartPar
Guillaume Croes,
Renaud Puybaret,
Janusz Bogdanowicz,
Umberto Celano,
Robert Gehlhaar,
Jan Genoe,
\sphinxstylestrong{Photonic Metamaterial with a Subwavelength Electrode Pattern},
Applied Optics 62,F14 (March 2023),
\sphinxhref{http://dx.doi.org/10.1364/AO.481396}{DOI: 10.1364/AO.481396}

\item {} 
\sphinxAtStartPar
Guillaume Croes,
Tsang\sphinxhyphen{}Hsuan Wang,
Robert Gehlhaar,
Jan Genoe,
\sphinxstylestrong{Sub\sphinxhyphen{}Wavelength Custom Wavefront Shaping by a Non\sphinxhyphen{}Linear Electro\sphinxhyphen{}Optic Spatial Light Modulator},
ACS Photonics 11,  pp. 529–536, Feb. 2024,
\sphinxhref{http://dx.doi.org/10.1021/acsphotonics.3c01401}{DOI: 10.1021/acsphotonics.3c01401}

\item {} 
\sphinxAtStartPar
Guillaume Croes,
V. Krasnov,
Robert Gehlhaar,
Jan Genoe,
\sphinxstylestrong{Computer Generated Holography for Waveguide based Holographic Displays},
Manuscript in preparation

\end{itemize}


\section{Conferences}
\label{\detokenize{Publications2:conferences}}\label{\detokenize{Publications2:conferences-target}}\begin{itemize}
\item {} 
\sphinxAtStartPar
Artur Hermans,
Robby Janneck,
Cedric Rolin,
S. Clemmen,
Paul Heremans,
Jan Genoe,
Roel Baets,
\sphinxstylestrong{Growth of Thin Film Organic Crystals with Strong Nonlinearity for On\sphinxhyphen{}Chip Second\sphinxhyphen{}Order Nonlinear Optics},
Proc. IEEE Photonics Benelux Symposium, Brussels, Belgium, November 15\sphinxhyphen{}16, 2018.

\item {} 
\sphinxAtStartPar
Guillaume Croes,
Nikolay Smolentsev,
Tsang\sphinxhyphen{}Hsuan Wang,
Robert Gehlhaar,
Jan Genoe,
\sphinxstylestrong{Non\sphinxhyphen{}linear electro\sphinxhyphen{}optic modelling of a Barium Titanate grating coupler},
Proc. SPIE 11484, 114840D: Optical Modeling and Performance Predictions XI (August 2020),
\sphinxhref{http://dx.doi.org/10.1117/12.2568032}{DOI: 10.1117/12.2568032}

\item {} 
\sphinxAtStartPar
Guillaume Croes,
Robert Gehlhaar,
Jan Genoe,
\sphinxstylestrong{Hologram Wavefront Shaping by a Non\sphinxhyphen{}Linear Electro\sphinxhyphen{}Optic Spatial Light Modulator},
Holography: Advances and Modern Trends VIII, April 2023, Prague, Czech Republic

\item {} 
\sphinxAtStartPar
Guillaume Croes,
Robert Gehlhaar,
Jan Genoe,
\sphinxstylestrong{Sub\sphinxhyphen{}Wavelength Custom Reprogrammable Active Photonic Platform for High\sphinxhyphen{}Resolution Beam Shaping and Holography},
Proc. SPIE PC12196, PC1219619: Active Photonic Platforms, San Diego, California, United States (October 2022)

\item {} 
\sphinxAtStartPar
Clement Merckling,
Islam Ahmed,
Tsang\sphinxhyphen{}Hsuan Wang,
Moloud Kaviani,
Jan Genoe,
Stefan De Gendt,
\sphinxstylestrong{Integrated Perovskites Oxides on Silicon: From Optical to Quantum Applications},
ECS Meeting Abstracts MA2022\sphinxhyphen{}01, 1060 , July 2022,
\sphinxhref{http://dx.doi.org/10.1149/MA2022-01191060mtgabs}{DOI: 10.1149/MA2022\sphinxhyphen{}01191060mtgabs}

\item {} 
\sphinxAtStartPar
Tsang\sphinxhyphen{}Hsuan Wang,
Robert Gehlhaar,
Thierry Conard,
Jan Genoe,
Clement Merckling,
\sphinxstylestrong{Interface Control and Characterization of SrTiO3/Si(001)},
Proc. E\sphinxhyphen{}MRS\sphinxhyphen{}fall, 20th to 23rd September 2021

\item {} 
\sphinxAtStartPar
Tsang\sphinxhyphen{}Hsuan Wang,
M. Korytov,
P. C. Hsu,
Jan Genoe, and
Clement Merckling,
\sphinxstylestrong{Single Crystalline BaTiO3 Grown by Pulsed\sphinxhyphen{}laser deposition (PLD) on SrTiO3 / Si Pseudo\sphinxhyphen{}substrate},
in Proc. E\sphinxhyphen{}MRS  spring, in Advanced functional films grown by pulsed deposition methods. Strasbourg, France, May 2020. \sphinxhref{https://www.european-mrs.com/advanced-functional-films-grown-pulsed-deposition-methods-emrs}{Online}

\end{itemize}


\section{PhD thesis}
\label{\detokenize{Publications2:phd-thesis}}\label{\detokenize{Publications2:thesis-target}}\begin{itemize}
\item {} 
\sphinxAtStartPar
Tsang\sphinxhyphen{}Hsuan Wang,
\sphinxstylestrong{Study of Barium Titanate Epitaxy on Silicon toward Its Application in Video Holography},
PhD Thesis, KULeuven, Leuven, Belgium, Monday, February 13, 2023.

\item {} 
\sphinxAtStartPar
Guillaume Croes,
\sphinxstylestrong{Subwavelength Barium Titanate Pockels modulation through transparent conductive oxide nanopillars: Exploring models for hologram construction from evanescent fields},
PhD Thesis, KULeuven, Leuven, Belgium, March 19,2025.

\end{itemize}

\sphinxstepscope

\begin{sphinxthebibliography}{100}
\bibitem[1]{bib:id9}
\sphinxAtStartPar
Guillaume Croes, Renaud Puybaret, Janusz Bogdanowicz, Umberto Celano, Robert Gehlhaar, and Jan Genoe. Photonic metamaterial with a subwavelength electrode pattern. \sphinxstyleemphasis{Applied Optics}, 62(17):F14–F20, June 2023. \sphinxhref{https://doi.org/10.1364/AO.481396}{doi:10.1364/AO.481396}.
\bibitem[2]{bib:id8}
\sphinxAtStartPar
Guillaume Croes, Nicolae Smolentsev, Tsang Hsuan Wang, Robert Gehlhaar, and Jan Genoe. Non\sphinxhyphen{}linear electro\sphinxhyphen{}optic modelling of a Barium Titanate grating coupler. In \sphinxstyleemphasis{Proc SPIE :Optical Modeling and Performance Predictions XI}, volume 11484, 114840D. Online Only, United States, August 2020. SPIE. \sphinxhref{https://doi.org/10.1117/12.2568032}{doi:10.1117/12.2568032}.
\bibitem[3]{bib:id6}
\sphinxAtStartPar
Guillaume Croes, Robert Gehlhaar, and Jan Genoe. Sub\sphinxhyphen{}wavelength custom reprogrammable active photonic platform for high\sphinxhyphen{}resolution beam shaping and holography. In \sphinxstyleemphasis{Active Photonic Platforms 2022}, volume PC12196, PC1219619. San Diego, California, United States, October 2022. SPIE. \sphinxhref{https://doi.org/10.1117/12.2632022}{doi:10.1117/12.2632022}.
\bibitem[4]{bib:id10}
\sphinxAtStartPar
Clement Merckling, Islam Ahmed, Tsang Hsuan Tsang, Moloud Kaviani, Jan Genoe, and Stefan De Gendt. (Invited) Integrated Perovskites Oxides on Silicon: From Optical to Quantum Applications. \sphinxstyleemphasis{ECS Meeting Abstracts}, MA2022\sphinxhyphen{}01(19):1060, July 2022. \sphinxhref{https://doi.org/10.1149/MA2022-01191060mtgabs}{doi:10.1149/MA2022\sphinxhyphen{}01191060mtgabs}.
\bibitem[5]{bib:id13}
\sphinxAtStartPar
Tsang\sphinxhyphen{}Hsuan Wang, Po\sphinxhyphen{}Chun (Brent) Hsu, Maxim Korytov, Jan Genoe, and Clement Merckling. Polarization control of epitaxial barium titanate (BaTiO3) grown by pulsed\sphinxhyphen{}laser deposition on a MBE\sphinxhyphen{}SrTiO3/Si(001) pseudo\sphinxhyphen{}substrate. \sphinxstyleemphasis{Journal of Applied Physics}, 128(10):104104, September 2020. \sphinxhref{https://doi.org/10.1063/5.0019980}{doi:10.1063/5.0019980}.
\bibitem[6]{bib:id14}
\sphinxAtStartPar
Tsang Hsuan Wang, M. Korytov, P. C. Hsu, Jan Genoe, and Clement Merckling. Single Crystalline BaTiO3 Grown by Pulsed\sphinxhyphen{}laser deposition (PLD) on SrTiO3 / Si Pseudo\sphinxhyphen{}substrate. In \sphinxstyleemphasis{Proc. E\sphinxhyphen{}MRS Spring}, Advanced Functional Films Grown by Pulsed Deposition Methods. Strasbourg, France, May 2020.
\bibitem[7]{bib:id12}
\sphinxAtStartPar
Tsang\sphinxhyphen{}Hsuan Wang, Robert Gehlhaar, Thierry Conard, Paola Favia, Jan Genoe, and Clement Merckling. Interfacial control of SrTiO3/Si(001) epitaxy and its effect on physical and optical properties. \sphinxstyleemphasis{Journal of Crystal Growth}, 582:126524, March 2022. \sphinxhref{https://doi.org/10.1016/j.jcrysgro.2022.126524}{doi:10.1016/j.jcrysgro.2022.126524}.
\bibitem[8]{bib:id11}
\sphinxAtStartPar
T\sphinxhyphen{}H Wang, Robert Gehlhaar, T. Conard, Jan Genoe, and Clement Merckling. Interface Control and Characterization of SrTiO3/Si(001). In \sphinxstyleemphasis{Proc. E\sphinxhyphen{}MRS\sphinxhyphen{}fall}. online Only, 20th to 23rd September 2021. MRS.
\bibitem[9]{bib:id3}
\sphinxAtStartPar
Tsang\sphinxhyphen{}Hsuan Wang, Jan Genoe, and Clement Merckling. \sphinxstyleemphasis{Study of Barium Titanate Epitaxy on Silicon toward Its Application in Video Holography}. PhD thesis, KULeuven, Leuven, Belgium, Monday, Feb 13, 2023 @17h00.
\bibitem[10]{bib:id331}
\sphinxAtStartPar
F. S. Chen. Optically Induced Change of Refractive Indices in LiNbO3 and LiTaO3. \sphinxstyleemphasis{Journal of Applied Physics}, 40(8):3389–3396, 7 1969. \sphinxhref{https://doi.org/10.1063/1.1658195}{doi:10.1063/1.1658195}.
\bibitem[11]{bib:id333}
\sphinxAtStartPar
A. Ashkin, G. D. Boyd, J. M. Dziedzic, R. G. Smith, A. A. Ballman, J. J. Levinstein, and K. Nassau. Optically\sphinxhyphen{}induced refractive index inhomogeneities in LiNbO3 and LiTaO3. \sphinxstyleemphasis{Applied Physics Letters}, 9(1):72–74, 1966. \sphinxhref{https://doi.org/10.1063/1.1754607}{doi:10.1063/1.1754607}.
\bibitem[12]{bib:id233}
\sphinxAtStartPar
F. S. Chen, J. T. LaMacchia, and D. B. Fraser. Holographic storage in Lithium Niobate. \sphinxstyleemphasis{Applied Physics Letters}, 13(7):223–225, 10 1968. \sphinxhref{https://doi.org/10.1063/1.1652580}{doi:10.1063/1.1652580}.
\bibitem[13]{bib:id232}
\sphinxAtStartPar
Deanna McMillen, Tracy Hudson, Julie Wagner, and Jere Singleton. Holographic recording in specially doped lithium niobate crystals. \sphinxstyleemphasis{Optics Express}, 2(12):491, 6 1998. \sphinxhref{https://doi.org/10.1364/OE.2.000491}{doi:10.1364/OE.2.000491}.
\bibitem[14]{bib:id311}
\sphinxAtStartPar
K. Buse, A. Adibi, and D. Psaltis. Non\sphinxhyphen{}volatile holographic storage in doubly doped lithium niobate crystals. \sphinxstyleemphasis{Nature}, 393(6686):665–668, 6 1998. \sphinxhref{https://doi.org/10.1038/31429}{doi:10.1038/31429}.
\bibitem[15]{bib:id94}
\sphinxAtStartPar
Fai H. Mok. Angle\sphinxhyphen{}multiplexed storage of 5000 holograms in lithium niobate. \sphinxstyleemphasis{Optics Letters}, 18(11):915, 6 1993. \sphinxhref{https://doi.org/10.1364/OL.18.000915}{doi:10.1364/OL.18.000915}.
\bibitem[16]{bib:id271}
\sphinxAtStartPar
M.N. Ernstoff, A.M. Leupp, M.J. Little, and H.T. Peterson. Liquid crystal pictorial display. In \sphinxstyleemphasis{1973 International Electron Devices Meeting}, 548–551. IRE, 1973. \sphinxhref{https://doi.org/10.1109/IEDM.1973.188783}{doi:10.1109/IEDM.1973.188783}.
\bibitem[17]{bib:id177}
\sphinxAtStartPar
N. Collings, W. A. Crossland, P. J. Ayliffe, D. G. Vass, and I. Underwood. Evolutionary development of advanced liquid crystal spatial light modulators. \sphinxstyleemphasis{Applied Optics}, 28(22):4740, 11 1989. \sphinxhref{https://doi.org/10.1364/AO.28.004740}{doi:10.1364/AO.28.004740}.
\bibitem[18]{bib:id393}
\sphinxAtStartPar
Kristina M. Johnson, Douglas J. McKnight, and Ian Underwood. Smart Spatial Light Modulators Using Liquid Crystals on Silicon. \sphinxstyleemphasis{IEEE Journal of Quantum Electronics}, 29(2):699–714, 1993. \sphinxhref{https://doi.org/10.1109/3.199323}{doi:10.1109/3.199323}.
\bibitem[19]{bib:id218}
\sphinxAtStartPar
D. P. Resler, D. S. Hobbs, R. C. Sharp, L. J. Friedman, and T. A. Dorschner. High\sphinxhyphen{}efficiency liquid\sphinxhyphen{}crystal optical phased\sphinxhyphen{}array beam steering. \sphinxstyleemphasis{Optics Letters}, 21(9):689, 5 1996. \sphinxhref{https://doi.org/10.1364/ol.21.000689}{doi:10.1364/ol.21.000689}.
\bibitem[20]{bib:id188}
\sphinxAtStartPar
Stephan Reichelt, Ralf Häussler, Gerald Fütterer, Norbert Leister, Hiromi Kato, Naru Usukura, and Yuuichi Kanbayashi. Full\sphinxhyphen{}range, complex spatial light modulator for real\sphinxhyphen{}time holography. \sphinxstyleemphasis{Optics Letters}, 37(11):1955, 6 2012. \sphinxhref{https://doi.org/10.1364/OL.37.001955}{doi:10.1364/OL.37.001955}.
\bibitem[21]{bib:id270}
\sphinxAtStartPar
David Vettese. Liquid crystal on silicon. \sphinxstyleemphasis{Nature Photonics 2010 4:11}, 4(11):752–754, 2010. \sphinxhref{https://doi.org/10.1038/nphoton.2010.252}{doi:10.1038/nphoton.2010.252}.
\bibitem[22]{bib:id191}
\sphinxAtStartPar
Zichen Zhang, Zheng You, and Daping Chu. Fundamentals of phase\sphinxhyphen{}only liquid crystal on silicon (LCOS) devices. \sphinxstyleemphasis{Light: Science \& Applications 2014 3:10}, 3(10):e213–e213, 10 2014. \sphinxhref{https://doi.org/10.1038/lsa.2014.94}{doi:10.1038/lsa.2014.94}.
\bibitem[23]{bib:id165}
\sphinxAtStartPar
Manuel Decker, Christian Kremers, Alexander Minovich, Isabelle Staude, Andrey E. Miroshnichenko, Dmitry Chigrin, Dragomir N. Neshev, Chennupati Jagadish, and Yuri S. Kivshar. Electro\sphinxhyphen{}optical switching by liquid\sphinxhyphen{}crystal controlled metasurfaces. \sphinxstyleemphasis{Optics Express}, 21(7):8879, 4 2013. \sphinxhref{https://doi.org/10.1364/OE.21.008879}{doi:10.1364/OE.21.008879}.
\bibitem[24]{bib:id143}
\sphinxAtStartPar
Maxim V. Gorkunov, Alena V. Mamonova, Irina V. Kasyanova, Alexander A. Ezhov, Vladimir V. Artemov, Ivan V. Simdyankin, and Artur R. Geivandov. Double\sphinxhyphen{}sided liquid crystal metasurfaces for electrically and mechanically controlled broadband visible anomalous refraction. \sphinxstyleemphasis{Nanophotonics}, 11(17):3901–3912, 9 2022. \sphinxhref{https://doi.org/10.1515/NANOPH-2022-0091}{doi:10.1515/NANOPH\sphinxhyphen{}2022\sphinxhyphen{}0091}.
\bibitem[25]{bib:id308}
\sphinxAtStartPar
J. B. Pendry. Negative Refraction Makes a Perfect Lens. \sphinxstyleemphasis{Physical Review Letters}, 85(18):3966–3969, 10 2000. \sphinxhref{https://doi.org/10.1103/PhysRevLett.85.3966}{doi:10.1103/PhysRevLett.85.3966}.
\bibitem[26]{bib:id178}
\sphinxAtStartPar
R. A. Shelby, D. R. Smith, and S. Schultz. Experimental Verification of a Negative Index of Refraction. \sphinxstyleemphasis{Science}, 292(5514):77–79, 4 2001. \sphinxhref{https://doi.org/10.1126/science.1058847}{doi:10.1126/science.1058847}.
\bibitem[27]{bib:id49}
\sphinxAtStartPar
N. I. Landy, S. Sajuyigbe, J. J. Mock, D. R. Smith, and W. J. Padilla. A Perfect Metamaterial Absorber. \sphinxstyleemphasis{Physical Review Letters}, 100(20):207402, 3 2008. \sphinxhref{https://doi.org/10.1103/PhysRevLett.100.207402}{doi:10.1103/PhysRevLett.100.207402}.
\bibitem[28]{bib:id136}
\sphinxAtStartPar
Amir Arbabi, Yu Horie, Mahmood Bagheri, and Andrei Faraon. Dielectric metasurfaces for complete control of phase and polarization with subwavelength spatial resolution and high transmission. \sphinxstyleemphasis{Nature Nanotechnology}, 10(11):937–943, 2015. \sphinxhref{https://doi.org/10.1038/nnano.2015.186}{doi:10.1038/nnano.2015.186}.
\bibitem[29]{bib:id268}
\sphinxAtStartPar
Zhenhe Ma, Xianghe Meng, Xiaodi Liu, Guangyuan Si, and Yan Jun Liu. Liquid Crystal Enabled Dynamic Nanodevices. \sphinxstyleemphasis{Nanomaterials 2018, Vol. 8, Page 871}, 8(11):871, 10 2018. \sphinxhref{https://doi.org/10.3390/NANO8110871}{doi:10.3390/NANO8110871}.
\bibitem[30]{bib:id449}
\sphinxAtStartPar
Arash Nemati, Qian Wang, Minghui Hong, and Jinghua Teng. Tunable and reconfigurable metasurfaces and metadevices. \sphinxstyleemphasis{Opto\sphinxhyphen{}Electronic Advances}, 1(5):18000901–18000925, 2018. \sphinxhref{https://doi.org/10.29026/oea.2018.180009}{doi:10.29026/oea.2018.180009}.
\bibitem[31]{bib:id451}
\sphinxAtStartPar
Tong Cui, Benfeng Bai, and Hong Bo Sun. Tunable Metasurfaces Based on Active Materials. \sphinxstyleemphasis{Advanced Functional Materials}, 3 2019. \sphinxhref{https://doi.org/10.1002/ADFM.201806692}{doi:10.1002/ADFM.201806692}.
\bibitem[32]{bib:id396}
\sphinxAtStartPar
Amr M. Shaltout, Vladimir M. Shalaev, and Mark L. Brongersma. Spatiotemporal light control with active metasurfaces. \sphinxstyleemphasis{Science}, 2019. \sphinxhref{https://doi.org/10.1126/SCIENCE.AAT3100}{doi:10.1126/SCIENCE.AAT3100}.
\bibitem[33]{bib:id454}
\sphinxAtStartPar
Chang Won Lee, Hee Jin Choi, and Heejeong Jeong. Tunable metasurfaces for visible and SWIR applications. \sphinxstyleemphasis{Nano Convergence}, 7(1):1–11, 12 2020. \sphinxhref{https://doi.org/10.1186/S40580-019-0213-2}{doi:10.1186/S40580\sphinxhyphen{}019\sphinxhyphen{}0213\sphinxhyphen{}2}.
\bibitem[34]{bib:id30}
\sphinxAtStartPar
Ruizhe Zhao, Lingling Huang, and Yongtian Wang. Recent advances in multi\sphinxhyphen{}dimensional metasurfaces holographic technologies. \sphinxstyleemphasis{PhotoniX}, 12 2020. \sphinxhref{https://doi.org/10.1186/s43074-020-00020-y}{doi:10.1186/s43074\sphinxhyphen{}020\sphinxhyphen{}00020\sphinxhyphen{}y}.
\bibitem[35]{bib:id56}
\sphinxAtStartPar
Jie Hu, Sankhyabrata Bandyopadhyay, Yu\sphinxhyphen{}hui Liu, and Li\sphinxhyphen{}yang Shao. A Review on Metasurface: From Principle to Smart Metadevices. \sphinxstyleemphasis{Frontiers in Physics}, 8:502, 1 2021. \sphinxhref{https://doi.org/10.3389/fphy.2020.586087}{doi:10.3389/fphy.2020.586087}.
\bibitem[36]{bib:id386}
\sphinxAtStartPar
Xiaoguang Zhao, Zhenci Sun, Lingyun Zhang, Zilun Wang, Rongbo Xie, Jiahao Zhao, Rui You, and Zheng You. Review on Metasurfaces: An Alternative Approach to Advanced Devices and Instruments. \sphinxstyleemphasis{Advanced Devices \& Instrumentation}, 2022:1–19, 9 2022. \sphinxhref{https://doi.org/10.34133/2022/9765089}{doi:10.34133/2022/9765089}.
\bibitem[37]{bib:id305}
\sphinxAtStartPar
I. C. Khoo, D. H. Werner, X. Liang, A. Diaz, and B. Weiner. Nanosphere dispersed liquid crystals for tunable negative\sphinxhyphen{}zero\sphinxhyphen{}positive index of refraction in the optical and terahertz regimes. \sphinxstyleemphasis{Optics Letters}, 31(17):2592, 9 2006. \sphinxhref{https://doi.org/10.1364/OL.31.002592}{doi:10.1364/OL.31.002592}.
\bibitem[38]{bib:id159}
\sphinxAtStartPar
Qian Zhao, Lei Kang, Bo Du, Bo Li, Ji Zhou, Hong Tang, Xiao Liang, and Baizhe Zhang. Electrically tunable negative permeability metamaterials based on nematic liquid crystals. \sphinxstyleemphasis{Applied Physics Letters}, 90(1):11112, 1 2007. \sphinxhref{https://doi.org/10.1063/1.2430485/332486}{doi:10.1063/1.2430485/332486}.
\bibitem[39]{bib:id23}
\sphinxAtStartPar
Bernhard Atorf, Holger Mühlenbernd, Mulda Muldarisnur, Thomas Zentgraf, and Heinz Kitzerow. Electro\sphinxhyphen{}optic tuning of split ring resonators embedded in a liquid crystal. \sphinxstyleemphasis{Optics Letters}, 39(5):1129, 3 2014. \sphinxhref{https://doi.org/10.1364/OL.39.001129}{doi:10.1364/OL.39.001129}.
\bibitem[40]{bib:id83}
\sphinxAtStartPar
Bernhard Atorf, Holger Mühlenbernd, Thomas Zentgraf, and Heinz Kitzerow. All\sphinxhyphen{}optical switching of a dye\sphinxhyphen{}doped liquid crystal plasmonic metasurface. \sphinxstyleemphasis{Optics Express}, 28(6):8898, 3 2020. \sphinxhref{https://doi.org/10.1364/OE.383877}{doi:10.1364/OE.383877}.
\bibitem[41]{bib:id157}
\sphinxAtStartPar
Mukesh Sharma, Netta Hendler, and Tal Ellenbogen. Electrically Switchable Color Tags Based on Active Liquid‐Crystal Plasmonic Metasurface Platform. \sphinxstyleemphasis{Advanced Optical Materials}, 8(7):1901182, 4 2020. \sphinxhref{https://doi.org/10.1002/adom.201901182}{doi:10.1002/adom.201901182}.
\bibitem[42]{bib:id112}
\sphinxAtStartPar
James A. Dolan, Haogang Cai, Lily Delalande, Xiao Li, Alex B. F. Martinson, Juan J. de Pablo, Daniel López, and Paul F. Nealey. Broadband Liquid Crystal Tunable Metasurfaces in the Visible: Liquid Crystal Inhomogeneities Across the Metasurface Parameter Space. \sphinxstyleemphasis{ACS Photonics}, 8(2):567–575, 2 2021. \sphinxhref{https://doi.org/10.1021/acsphotonics.0c01599}{doi:10.1021/acsphotonics.0c01599}.
\bibitem[43]{bib:id450}
\sphinxAtStartPar
Alexander Minovich, Dragomir N. Neshev, David A. Powell, Ilya V. Shadrivov, and Yuri S. Kivshar. Tunable fishnet metamaterials infiltrated by liquid crystals. \sphinxstyleemphasis{Applied Physics Letters}, 4 2010. \sphinxhref{https://doi.org/10.1063/1.3427429}{doi:10.1063/1.3427429}.
\bibitem[44]{bib:id121}
\sphinxAtStartPar
Yibo Ni, Chen Chen, Shun Wen, Xinyuan Xue, Liqun Sun, and Yuanmu Yang. Computational spectropolarimetry with a tunable liquid crystal metasurface. \sphinxstyleemphasis{eLight}, 2(1):1–10, 12 2022. \sphinxhref{https://doi.org/10.1186/S43593-022-00032-0/FIGURES/4}{doi:10.1186/S43593\sphinxhyphen{}022\sphinxhyphen{}00032\sphinxhyphen{}0/FIGURES/4}.
\bibitem[45]{bib:id227}
\sphinxAtStartPar
Seung\sphinxhyphen{}Yeol Lee, Yong\sphinxhyphen{}Hae Kim, Seong\sphinxhyphen{}M. Cho, Gi Heon Kim, Tae\sphinxhyphen{}Youb Kim, Hojun Ryu, Han Na Kim, Han Byeol Kang, Chi\sphinxhyphen{}Young Hwang, and Chi\sphinxhyphen{}Sun Hwang. Holographic image generation with a thin\sphinxhyphen{}film resonance caused by chalcogenide phase\sphinxhyphen{}change material. \sphinxstyleemphasis{Scientific Reports}, 7(1):41152, 1 2017. \sphinxhref{https://doi.org/10.1038/srep41152}{doi:10.1038/srep41152}.
\bibitem[46]{bib:id17}
\sphinxAtStartPar
Jianxiong Li, Simon Kamin, Guoxing Zheng, Frank Neubrech, Shuang Zhang, and Na Liu. Addressable metasurfaces for dynamic holography and optical information encryption. \sphinxstyleemphasis{Science Advances}, 6 2018. \sphinxhref{https://doi.org/10.1126/sciadv.aar6768}{doi:10.1126/sciadv.aar6768}.
\bibitem[47]{bib:id332}
\sphinxAtStartPar
Qian Wang, Edward T. F. Rogers, Behrad Gholipour, Chih\sphinxhyphen{}Ming Wang, Guanghui Yuan, Jinghua Teng, and Nikolay I. Zheludev. Optically reconfigurable metasurfaces and photonic devices based on phase change materials. \sphinxstyleemphasis{Nature Photonics}, 10(1):60–65, 1 2016. \sphinxhref{https://doi.org/10.1038/nphoton.2015.247}{doi:10.1038/nphoton.2015.247}.
\bibitem[48]{bib:id417}
\sphinxAtStartPar
Jing Zhao, Chunmei Ouyang, Xieyu Chen, Yanfeng Li, Caihong Zhang, Longcheng Feng, Biaobing Jin, Jiajun Ma, Yi Liu, Shoujun Zhang, Quan Xu, Jiaguang Han, and Weili Zhang. Temperature\sphinxhyphen{}controlled terahertz polarization conversion bandwidth. \sphinxstyleemphasis{Optics Express}, 29(14):21738, 7 2021. \sphinxhref{https://doi.org/10.1364/OE.431622}{doi:10.1364/OE.431622}.
\bibitem[49]{bib:id372}
\sphinxAtStartPar
Ling Wang, Weijun Hong, Li Deng, Shufang Li, Chen Zhang, Jianfeng Zhu, and Hongjun Wang. Reconfigurable Multifunctional Metasurface Hybridized with Vanadium Dioxide at Terahertz Frequencies. \sphinxstyleemphasis{Materials}, 11(10):2040, 10 2018. \sphinxhref{https://doi.org/10.3390/ma11102040}{doi:10.3390/ma11102040}.
\bibitem[50]{bib:id374}
\sphinxAtStartPar
Alexej V. Pogrebnyakov, Jeremy A. Bossard, Jeremiah P. Turpin, J. David Musgraves, Hee Jung Shin, Clara Rivero\sphinxhyphen{}Baleine, Nikolas Podraza, Kathleen A. Richardson, Douglas H. Werner, and Theresa S. Mayer. Reconfigurable near\sphinxhyphen{}IR metasurface based on Ge <sub>2</sub> Sb <sub>2</sub> Te <sub>5</sub> phase\sphinxhyphen{}change material. \sphinxstyleemphasis{Optical Materials Express}, 8(8):2264, 8 2018. \sphinxhref{https://doi.org/10.1364/OME.8.002264}{doi:10.1364/OME.8.002264}.
\bibitem[51]{bib:id62}
\sphinxAtStartPar
Jiajia Chen, Xieyu Chen, Kuan Liu, Shoujun Zhang, Tun Cao, and Zhen Tian. A Thermally Switchable Bifunctional Metasurface for Broadband Polarization Conversion and Absorption Based on Phase‐Change Material. \sphinxstyleemphasis{Advanced Photonics Research}, pages 2100369, 4 2022. \sphinxhref{https://doi.org/10.1002/ADPR.202100369}{doi:10.1002/ADPR.202100369}.
\bibitem[52]{bib:id474}
\sphinxAtStartPar
Ann\sphinxhyphen{}Katrin U. Michel, Dmitry N. Chigrin, Tobias W. W. Maß, Kathrin Schönauer, Martin Salinga, Matthias Wuttig, and Thomas Taubner. Using Low\sphinxhyphen{}Loss Phase\sphinxhyphen{}Change Materials for Mid\sphinxhyphen{}Infrared Antenna Resonance Tuning. \sphinxstyleemphasis{Nano Letters}, 13(8):3470–3475, 8 2013. \sphinxhref{https://doi.org/10.1021/nl4006194}{doi:10.1021/nl4006194}.
\bibitem[53]{bib:id173}
\sphinxAtStartPar
Yiguo Chen, Xiong Li, Yannick Sonnefraud, Antonio I. Fernández\sphinxhyphen{}Domínguez, Xiangang Luo, Minghui Hong, and Stefan A. Maier. Engineering the Phase Front of Light with Phase\sphinxhyphen{}Change Material Based Planar lenses. \sphinxstyleemphasis{Scientific Reports}, 5(1):8660, 3 2015. \sphinxhref{https://doi.org/10.1038/srep08660}{doi:10.1038/srep08660}.
\bibitem[54]{bib:id155}
\sphinxAtStartPar
Sajjad Abdollahramezani, Omid Hemmatyar, Mohammad Taghinejad, Hossein Taghinejad, Alex Krasnok, Ali A. Eftekhar, Christian Teichrib, Sanchit Deshmukh, Mostafa A. El\sphinxhyphen{}Sayed, Eric Pop, Matthias Wuttig, Andrea Alù, Wenshan Cai, and Ali Adibi. Electrically driven reprogrammable phase\sphinxhyphen{}change metasurface reaching 80\% efficiency. \sphinxstyleemphasis{Nature Communications}, 13(1):1696, 3 2022. \sphinxhref{https://doi.org/10.1038/s41467-022-29374-6}{doi:10.1038/s41467\sphinxhyphen{}022\sphinxhyphen{}29374\sphinxhyphen{}6}.
\bibitem[55]{bib:id148}
\sphinxAtStartPar
M. Tayyab Nouman, Jihyun Hwang, Mohd. Faiyaz, Gyejung Lee, Do\sphinxhyphen{}Young Noh, and Jae\sphinxhyphen{}Hyung Jang. Dynamic\sphinxhyphen{}Metasurface\sphinxhyphen{}Based Cavity Structures for Enhanced Absorption and Phase Modulation. \sphinxstyleemphasis{ACS Photonics}, 6(2):374–381, 2 2019. \sphinxhref{https://doi.org/10.1021/acsphotonics.8b01014}{doi:10.1021/acsphotonics.8b01014}.
\bibitem[56]{bib:id76}
\sphinxAtStartPar
Jürgen Sautter, Isabelle Staude, Manuel Decker, Evgenia Rusak, Dragomir N. Neshev, Igal Brener, and Yuri S. Kivshar. Active tuning of all\sphinxhyphen{}dielectric metasurfaces. \sphinxstyleemphasis{ACS Nano}, 9(4):4308–4315, 4 2015. \sphinxhref{https://doi.org/10.1021/ACSNANO.5B00723}{doi:10.1021/ACSNANO.5B00723}.
\bibitem[57]{bib:id69}
\sphinxAtStartPar
Cheng Hung Chu, Ming Lun Tseng, Jie Chen, Pin Chieh Wu, Yi Hao Chen, Hsiang Chu Wang, Ting Yu Chen, Wen Ting Hsieh, Hui Jun Wu, Greg Sun, and Din Ping Tsai. Active dielectric metasurface based on phase\sphinxhyphen{}change medium. \sphinxstyleemphasis{Laser \& Photonics Reviews}, 10(6):986–994, 11 2016. \sphinxhref{https://doi.org/10.1002/LPOR.201600106}{doi:10.1002/LPOR.201600106}.
\bibitem[58]{bib:id82}
\sphinxAtStartPar
Chenglong Zheng, Jie Li, Zhen Yue, Jitao Li, Jingyu Liu, Guocui Wang, Silei Wang, Yating Zhang, Yan Zhang, and Jianquan Yao. All\sphinxhyphen{}Dielectric Trifunctional Metasurface Capable of Independent Amplitude and Phase Modulation. \sphinxstyleemphasis{Laser \& Photonics Reviews}, 16(7):2200051, 7 2022. \sphinxhref{https://doi.org/10.1002/LPOR.202200051}{doi:10.1002/LPOR.202200051}.
\bibitem[59]{bib:id28}
\sphinxAtStartPar
Ming Zhang, Mingbo Pu, Fei Zhang, Yinghui Guo, Qiong He, Xiaoliang Ma, Yijia Huang, Xiong Li, Honglin Yu, and Xiangang Luo. Plasmonic Metasurfaces for Switchable Photonic Spin\sphinxhyphen{}Orbit Interactions Based on Phase Change Materials. \sphinxstyleemphasis{Advanced Science}, 5(10):1800835, 10 2018. \sphinxhref{https://doi.org/10.1002/advs.201800835}{doi:10.1002/advs.201800835}.
\bibitem[60]{bib:id387}
\sphinxAtStartPar
Chi\sphinxhyphen{}Young Hwang, Gi Heon Kim, Jong\sphinxhyphen{}Heon Yang, Chi\sphinxhyphen{}Sun Hwang, Seong M. Cho, Won\sphinxhyphen{}Jae Lee, Jae\sphinxhyphen{}Eun Pi, Ji Hun Choi, Kyunghee Choi, Hee\sphinxhyphen{}Ok Kim, Seung\sphinxhyphen{}Yeol Lee, and Yong\sphinxhyphen{}Hae Kim. Rewritable full\sphinxhyphen{}color computer\sphinxhyphen{}generated holograms based on color\sphinxhyphen{}selective diffractive optical components including phase\sphinxhyphen{}change materials. \sphinxstyleemphasis{Nanoscale}, 10(46):21648–21655, 2018. \sphinxhref{https://doi.org/10.1039/C8NR04471F}{doi:10.1039/C8NR04471F}.
\bibitem[61]{bib:id44}
\sphinxAtStartPar
Kaichen Dong, Sukjoon Hong, Yang Deng, He Ma, Jiachen Li, Xi Wang, Junyeob Yeo, Letian Wang, Shuai Lou, Kyle B. Tom, Kai Liu, Zheng You, Yang Wei, Costas P. Grigoropoulos, Jie Yao, and Junqiao Wu. A Lithography‐Free and Field‐Programmable Photonic Metacanvas. \sphinxstyleemphasis{Advanced Materials}, 2 2018. \sphinxhref{https://doi.org/10.1002/adma.201703878}{doi:10.1002/adma.201703878}.
\bibitem[62]{bib:id437}
\sphinxAtStartPar
Xingbo Liu, Qiu Wang, Xueqian Zhang, Hua Li, Quan Xu, Yuehong Xu, Xieyu Chen, Shaoxian Li, Meng Liu, Zhen Tian, Caihong Zhang, Chongwen Zou, Jiaguang Han, and Weili Zhang. Thermally Dependent Dynamic Meta‐Holography Using a Vanadium Dioxide Integrated Metasurface. \sphinxstyleemphasis{Advanced Optical Materials}, 7(12):1900175, 6 2019. \sphinxhref{https://doi.org/10.1002/adom.201900175}{doi:10.1002/adom.201900175}.
\bibitem[63]{bib:id412}
\sphinxAtStartPar
SWAVE Photonics. Swave Photonics Developing World’s First True Holographic Display Technology To Power Reality\sphinxhyphen{}First Spatial Computing. 2024. URL: \sphinxurl{https://swave.io/swave-photonics-developing-worlds-first-true-holographic-display-technology-to-power-reality-first-spatial-computing/}.
\bibitem[64]{bib:id330}
\sphinxAtStartPar
Jason Midkiff, Kyoung Min Yoo, Jong\sphinxhyphen{}Dug Shin, Hamed Dalir, Mohammad Teimourpour, and Ray T. Chen. Optical phased array beam steering in the mid\sphinxhyphen{}infrared on an InP\sphinxhyphen{}based platform. \sphinxstyleemphasis{Optica}, 7(11):1544, 11 2020. \sphinxhref{https://doi.org/10.1364/OPTICA.400441}{doi:10.1364/OPTICA.400441}.
\bibitem[65]{bib:id258}
\sphinxAtStartPar
Jie Sun, Erman Timurdogan, Ami Yaacobi, Ehsan Shah Hosseini, and Michael R. Watts. Large\sphinxhyphen{}scale nanophotonic phased array. \sphinxstyleemphasis{Nature}, 493(7431):195–199, 1 2013. \sphinxhref{https://doi.org/10.1038/nature11727}{doi:10.1038/nature11727}.
\bibitem[66]{bib:id496}
\sphinxAtStartPar
Hiroyuki Ito, Yuma Kusunoki, Jun Maeda, Daichi Akiyama, Naoya Kodama, Hiroshi Abe, Ryo Tetsuya, and Toshihiko Baba. Wide beam steering by slow\sphinxhyphen{}light waveguide gratings and a prism lens. \sphinxstyleemphasis{Optica}, 7(1):47, 1 2020. \sphinxhref{https://doi.org/10.1364/OPTICA.381484}{doi:10.1364/OPTICA.381484}.
\bibitem[67]{bib:id438}
\sphinxAtStartPar
Edward D. Palik. Chapter 3 \sphinxhyphen{} thermo\sphinxhyphen{}optic coefficients. In \sphinxstyleemphasis{Handbook of Optical Constants of Solids}, pages 115–261. Academic Press, Burlington, 1997. \sphinxhref{https://doi.org/10.1016/B978-012544415-6.50150-3}{doi:10.1016/B978\sphinxhyphen{}012544415\sphinxhyphen{}6.50150\sphinxhyphen{}3}.
\bibitem[68]{bib:id241}
\sphinxAtStartPar
Ami Yaacobi, Jie Sun, Michele Moresco, Gerald Leake, Douglas Coolbaugh, and Michael R. Watts. Integrated phased array for wide\sphinxhyphen{}angle beam steering. \sphinxstyleemphasis{Optics Letters}, 39(15):4575, 8 2014. \sphinxhref{https://doi.org/10.1364/OL.39.004575}{doi:10.1364/OL.39.004575}.
\bibitem[69]{bib:id131}
\sphinxAtStartPar
Daisuke Inoue, Tadashi Ichikawa, Akari Kawasaki, and Tatsuya Yamashita. Demonstration of a new optical scanner using silicon photonics integrated circuit. \sphinxstyleemphasis{Optics Express}, 27(3):2499, 2 2019. \sphinxhref{https://doi.org/10.1364/OE.27.002499}{doi:10.1364/OE.27.002499}.
\bibitem[70]{bib:id466}
\sphinxAtStartPar
Karel Van Acoleyen, Hendrik Rogier, and Roel Baets. Two\sphinxhyphen{}dimensional optical phased array antenna on silicon\sphinxhyphen{}on\sphinxhyphen{}Insulator. \sphinxstyleemphasis{Optics Express}, 18(13):13655, 6 2010. \sphinxhref{https://doi.org/10.1364/oe.18.013655}{doi:10.1364/oe.18.013655}.
\bibitem[71]{bib:id463}
\sphinxAtStartPar
William S. Rabinovich, Peter G. Goetz, Marcel W. Pruessner, Rita Mahon, Mike S. Ferraro, Doe Park, Erin Fleet, and Michael J. DePrenger. Two\sphinxhyphen{}dimensional beam steering using a thermo\sphinxhyphen{}optic silicon photonic optical phased array. \sphinxstyleemphasis{Optical Engineering}, 55(11):111603, 8 2016. \sphinxhref{https://doi.org/10.1117/1.OE.55.11.111603}{doi:10.1117/1.OE.55.11.111603}.
\bibitem[72]{bib:id439}
\sphinxAtStartPar
Seong\sphinxhyphen{}Hwan Kim, Jong\sphinxhyphen{}Bum You, Yun\sphinxhyphen{}Gi Ha, Geumbong Kang, Dae\sphinxhyphen{}Seong Lee, Hyeonho Yoon, Dong\sphinxhyphen{}Eun Yoo, Dong\sphinxhyphen{}Wook Lee, Kyoungsik Yu, Chan\sphinxhyphen{}Hyun Youn, and Hyo\sphinxhyphen{}Hoon Park. Thermo\sphinxhyphen{}optic control of the longitudinal radiation angle in a silicon\sphinxhyphen{}based optical phased array. \sphinxstyleemphasis{Optics Letters}, 44(2):411, 1 2019. \sphinxhref{https://doi.org/10.1364/OL.44.000411}{doi:10.1364/OL.44.000411}.
\bibitem[73]{bib:id263}
\sphinxAtStartPar
Chao Li, Xianyi Cao, Kan Wu, Xinwan Li, and Jianping Chen. Lens\sphinxhyphen{}based integrated 2D beam\sphinxhyphen{}steering device with defocusing approach and broadband pulse operation for Lidar application. \sphinxstyleemphasis{Optics Express}, 27(23):32970, 2019. \sphinxhref{https://doi.org/10.1364/OE.27.032970}{doi:10.1364/OE.27.032970}.
\bibitem[74]{bib:id264}
\sphinxAtStartPar
Xianyi Cao, Gaofeng Qiu, Kan Wu, Chao Li, and Jianping Chen. Lidar system based on lens assisted integrated beam steering. \sphinxstyleemphasis{Optics Letters}, 45(20):5816, 10 2020. \sphinxhref{https://doi.org/10.1364/OL.401486}{doi:10.1364/OL.401486}.
\bibitem[75]{bib:id262}
\sphinxAtStartPar
Kamran Qaderi and Daniel E. Smalley. Leaky\sphinxhyphen{}mode waveguide modulators with high deflection angle for use in holographic video displays. \sphinxstyleemphasis{Optics Express}, 24(18):20831, 9 2016. \sphinxhref{https://doi.org/10.1364/OE.24.020831}{doi:10.1364/OE.24.020831}.
\bibitem[76]{bib:id98}
\sphinxAtStartPar
Dorian Treptow, Raúl Bola, Estela Martín\sphinxhyphen{}Badosa, and Mario Montes\sphinxhyphen{}Usategui. Artifact\sphinxhyphen{}free holographic light shaping through moving acousto\sphinxhyphen{}optic holograms. \sphinxstyleemphasis{Scientific Reports}, 11(1):21261, 10 2021. \sphinxhref{https://doi.org/10.1038/s41598-021-00332-4}{doi:10.1038/s41598\sphinxhyphen{}021\sphinxhyphen{}00332\sphinxhyphen{}4}.
\bibitem[77]{bib:id18}
\sphinxAtStartPar
D. E. Smalley, Q. Y.J. Smithwick, V. M. Bove, J. Barabas, and S. Jolly. Anisotropic leaky\sphinxhyphen{}mode modulator for holographic video displays. \sphinxstyleemphasis{Nature}, 498(7454):313–317, 2013. \sphinxhref{https://doi.org/10.1038/nature12217}{doi:10.1038/nature12217}.
\bibitem[78]{bib:id210}
\sphinxAtStartPar
Valery V. Proklov and E. M. Korablev. Guided\sphinxhyphen{}wave multichannel acousto\sphinxhyphen{}optic devices based on collinear wave propagation. In Antoni Sliwinski, Piotr Kwiek, B. Linde, and A. Markiewicz, editors, \sphinxstyleemphasis{Acousto\sphinxhyphen{}Optics and Applications}, volume 1844, 112–125. SPIE, 11 1992. \sphinxhref{https://doi.org/10.1117/12.131919}{doi:10.1117/12.131919}.
\bibitem[79]{bib:id67}
\sphinxAtStartPar
U. Rust and E. Strake. Acoustooptical Coupling of Guided to Substrate Modes in Planar Proton\sphinxhyphen{}Exchanged LiNbO3\sphinxhyphen{}Waveguides. In \sphinxstyleemphasis{Integrated Photonics Research}, ME4. Washington, D.C., 4 1992. OSA. \sphinxhref{https://doi.org/10.1364/IPR.1992.ME4}{doi:10.1364/IPR.1992.ME4}.
\bibitem[80]{bib:id89}
\sphinxAtStartPar
Kenchi Ito and Kazumi Kawamoto. An optical deflector using collinear acoustooptic coupling fabricated on proton\sphinxhyphen{}exchanged LiNbO3. \sphinxstyleemphasis{Japanese Journal of Applied Physics, Part 1: Regular Papers and Short Notes and Review Papers}, 37(9 A):4858–4865, 9 1998. \sphinxhref{https://doi.org/10.1143/JJAP.37.4858/XML}{doi:10.1143/JJAP.37.4858/XML}.
\bibitem[81]{bib:id211}
\sphinxAtStartPar
C. S. Tsai, Q. Li, and C. L. Chang. Guided\sphinxhyphen{}wave two\sphinxhyphen{}dimensional acousto\sphinxhyphen{}optic scanner using proton\sphinxhyphen{}exchanged lithium niobate waveguide. \sphinxstyleemphasis{Fiber and Integrated Optics}, 17(3):157–166, 1998. \sphinxhref{https://doi.org/10.1080/014680398244902}{doi:10.1080/014680398244902}.
\bibitem[82]{bib:id399}
\sphinxAtStartPar
Daniel E. Smalley, Sundeep Jolly, Gregg E. Favalora, and Michael G. Moebius. Status of Leaky Mode Holography. \sphinxstyleemphasis{Photonics 2021, Vol. 8, Page 292}, 8(8):292, 7 2021. \sphinxhref{https://doi.org/10.3390/PHOTONICS8080292}{doi:10.3390/PHOTONICS8080292}.
\bibitem[83]{bib:id184}
\sphinxAtStartPar
Bingzhao Li, Qixuan Lin, and Mo Li. Frequency–angular resolving LiDAR using chip\sphinxhyphen{}scale acousto\sphinxhyphen{}optic beam steering. \sphinxstyleemphasis{Nature}, 620(7973):316–322, 8 2023. \sphinxhref{https://doi.org/10.1038/s41586-023-06201-6}{doi:10.1038/s41586\sphinxhyphen{}023\sphinxhyphen{}06201\sphinxhyphen{}6}.
\bibitem[84]{bib:id444}
\sphinxAtStartPar
David S. Ginley and Clark Bright. Transparent Conducting Oxides. \sphinxstyleemphasis{MRS Bulletin}, 25(8):15–18, 8 2000. \sphinxhref{https://doi.org/10.1557/mrs2000.256}{doi:10.1557/mrs2000.256}.
\bibitem[85]{bib:id446}
\sphinxAtStartPar
Elvira Fortunato, David Ginley, Hideo Hosono, and David C. Paine. Transparent Conducting Oxides for Photovoltaics. \sphinxstyleemphasis{MRS Bulletin}, 32(3):242–247, 3 2007. \sphinxhref{https://doi.org/10.1557/mrs2007.29}{doi:10.1557/mrs2007.29}.
\bibitem[86]{bib:id448}
\sphinxAtStartPar
Andreas Stadler. Transparent Conducting Oxides—An Up\sphinxhyphen{}To\sphinxhyphen{}Date Overview. \sphinxstyleemphasis{Materials}, 5(12):661–683, 4 2012. \sphinxhref{https://doi.org/10.3390/ma5040661}{doi:10.3390/ma5040661}.
\bibitem[87]{bib:id339}
\sphinxAtStartPar
Albert de Jamblinne de Meux, Ajay Bhoolokam, Geoffrey Pourtois, Jan Genoe, and Paul Heremans. Oxygen vacancies effects in a‐IGZO: Formation mechanisms, hysteresis, and negative bias stress effects. \sphinxstyleemphasis{physica status solidi (a)}, 214(6):1600889, 6 2017. \sphinxhref{https://doi.org/10.1002/pssa.201600889}{doi:10.1002/pssa.201600889}.
\bibitem[88]{bib:id341}
\sphinxAtStartPar
Lishu Liu, Zengxia Mei, Aihua Tang, Alexander Azarov, Andrej Kuznetsov, Qi\sphinxhyphen{}Kun Xue, and Xiaolong Du. Oxygen vacancies: The origin of n \sphinxhyphen{}type conductivity in ZnO. \sphinxstyleemphasis{Physical Review B}, 93(23):235305, 6 2016. \sphinxhref{https://doi.org/10.1103/PhysRevB.93.235305}{doi:10.1103/PhysRevB.93.235305}.
\bibitem[89]{bib:id338}
\sphinxAtStartPar
Gururaj V. Naik, Jongbum Kim, and Alexandra Boltasseva. Oxides and nitrides as alternative plasmonic materials in the optical range. \sphinxstyleemphasis{Optical Materials Express}, 1(6):1090, 10 2011. \sphinxhref{https://doi.org/10.1364/OME.1.001090}{doi:10.1364/OME.1.001090}.
\bibitem[90]{bib:id498}
\sphinxAtStartPar
Yu Wang, Antonio Capretti, and Luca Dal Negro. Wide tuning of the optical and structural properties of alternative plasmonic materials. \sphinxstyleemphasis{Optical Materials Express}, 5(11):2415, 11 2015. \sphinxhref{https://doi.org/10.1364/OME.5.002415}{doi:10.1364/OME.5.002415}.
\bibitem[91]{bib:id334}
\sphinxAtStartPar
H. Kim, M. Osofsky, S. M. Prokes, O. J. Glembocki, and A. Piqué. Optimization of Al\sphinxhyphen{}doped ZnO films for low loss plasmonic materials at telecommunication wavelengths. \sphinxstyleemphasis{Applied Physics Letters}, 102(17):171103, 4 2013. \sphinxhref{https://doi.org/10.1063/1.4802901}{doi:10.1063/1.4802901}.
\bibitem[92]{bib:id86}
\sphinxAtStartPar
Gururaj V. Naik, Vladimir M. Shalaev, and Alexandra Boltasseva. Alternative Plasmonic Materials: Beyond Gold and Silver. \sphinxstyleemphasis{Advanced Materials}, 25(24):3264–3294, 6 2013. \sphinxhref{https://doi.org/10.1002/adma.201205076}{doi:10.1002/adma.201205076}.
\bibitem[93]{bib:id445}
\sphinxAtStartPar
Viktoriia E. Babicheva, Alexandra Boltasseva, and Andrei V. Lavrinenko. Transparent conducting oxides for electro\sphinxhyphen{}optical plasmonic modulators. \sphinxstyleemphasis{Nanophotonics}, 4(1):165–185, 6 2015. \sphinxhref{https://doi.org/10.1515/NANOPH-2015-0004}{doi:10.1515/NANOPH\sphinxhyphen{}2015\sphinxhyphen{}0004}.
\bibitem[94]{bib:id447}
\sphinxAtStartPar
Wallace Jaffray, Soham Saha, Vladimir M. Shalaev, Alexandra Boltasseva, and Marcello Ferrera. Transparent conducting oxides: from all\sphinxhyphen{}dielectric plasmonics to a new paradigm in integrated photonics. \sphinxstyleemphasis{Advances in Optics and Photonics}, 14(2):148, 6 2022. \sphinxhref{https://doi.org/10.1364/AOP.448391}{doi:10.1364/AOP.448391}.
\bibitem[95]{bib:id472}
\sphinxAtStartPar
Eyal Feigenbaum, Kenneth Diest, and Harry A. Atwater. Unity\sphinxhyphen{}Order Index Change in Transparent Conducting Oxides at Visible Frequencies. \sphinxstyleemphasis{Nano Letters}, 10(6):2111–2116, 6 2010. \sphinxhref{https://doi.org/10.1021/nl1006307}{doi:10.1021/nl1006307}.
\bibitem[96]{bib:id71}
\sphinxAtStartPar
Soo Jin Kim and Mark L. Brongersma. Active flat optics using a guided mode resonance. \sphinxstyleemphasis{Optics Letters}, 42(1):5, 1 2017. \sphinxhref{https://doi.org/10.1364/OL.42.000005}{doi:10.1364/OL.42.000005}.
\bibitem[97]{bib:id84}
\sphinxAtStartPar
Junghyun Park, Byung Gil Jeong, Sun Il Kim, Duhyun Lee, Jungwoo Kim, Changgyun Shin, Chang Bum Lee, Tatsuhiro Otsuka, Jisoo Kyoung, Sangwook Kim, Ki\sphinxhyphen{}Yeon Yang, Yong\sphinxhyphen{}Young Park, Jisan Lee, Inoh Hwang, Jaeduck Jang, Seok Ho Song, Mark L. Brongersma, Kyoungho Ha, Sung\sphinxhyphen{}Woo Hwang, Hyuck Choo, and Byoung Lyong Choi. All\sphinxhyphen{}solid\sphinxhyphen{}state spatial light modulator with independent phase and amplitude control for three\sphinxhyphen{}dimensional LiDAR applications. \sphinxstyleemphasis{Nature Nanotechnology}, 16(1):69–76, 1 2021. \sphinxhref{https://doi.org/10.1038/s41565-020-00787-y}{doi:10.1038/s41565\sphinxhyphen{}020\sphinxhyphen{}00787\sphinxhyphen{}y}.
\bibitem[98]{bib:id468}
\sphinxAtStartPar
Zhaolin Lu, Wangshi Zhao, and Kaifeng Shi. Ultracompact electroabsorption modulators based on tunable epsilon\sphinxhyphen{}near\sphinxhyphen{}zero\sphinxhyphen{}slot waveguides. \sphinxstyleemphasis{IEEE Photonics Journal}, 4(3):735–740, 2012. \sphinxhref{https://doi.org/10.1109/JPHOT.2012.2197742}{doi:10.1109/JPHOT.2012.2197742}.
\bibitem[99]{bib:id158}
\sphinxAtStartPar
Junghyun Park, Ju\sphinxhyphen{}Hyung Kang, Xiaoge Liu, and Mark L. Brongersma. Electrically Tunable Epsilon\sphinxhyphen{}Near\sphinxhyphen{}Zero (ENZ) Metafilm Absorbers. \sphinxstyleemphasis{Scientific Reports}, 5(1):15754, 11 2015. \sphinxhref{https://doi.org/10.1038/srep15754}{doi:10.1038/srep15754}.
\bibitem[100]{bib:id349}
\sphinxAtStartPar
A. V. Krasavin and A. V. Zayats. Photonic Signal Processing on Electronic Scales: Electro\sphinxhyphen{}Optical Field\sphinxhyphen{}Effect Nanoplasmonic Modulator. \sphinxstyleemphasis{Physical Review Letters}, 109(5):053901, 7 2012. \sphinxhref{https://doi.org/10.1103/PhysRevLett.109.053901}{doi:10.1103/PhysRevLett.109.053901}.
\bibitem[101]{bib:id354}
\sphinxAtStartPar
Viktoriia E. Babicheva and Andrei V. Lavrinenko. Plasmonic modulator optimized by patterning of active layer and tuning permittivity. \sphinxstyleemphasis{Optics Communications}, 285(24):5500–5507, 11 2012. \sphinxhref{https://doi.org/10.1016/j.optcom.2012.07.117}{doi:10.1016/j.optcom.2012.07.117}.
\bibitem[102]{bib:id162}
\sphinxAtStartPar
Alok P. Vasudev, Ju\sphinxhyphen{}Hyung Kang, Junghyun Park, Xiaoge Liu, and Mark L. Brongersma. Electro\sphinxhyphen{}optical modulation of a silicon waveguide with an “epsilon\sphinxhyphen{}near\sphinxhyphen{}zero” material. \sphinxstyleemphasis{Optics Express}, 21(22):26387, 11 2013. \sphinxhref{https://doi.org/10.1364/OE.21.026387}{doi:10.1364/OE.21.026387}.
\bibitem[103]{bib:id304}
\sphinxAtStartPar
Ho W. Lee, Georgia Papadakis, Stanley P. Burgos, Krishnan Chander, Arian Kriesch, Ragip Pala, Ulf Peschel, and Harry A. Atwater. Nanoscale conducting oxide PlasMOStor. \sphinxstyleemphasis{Nano Letters}, 14(11):6463–6468, 11 2014. \sphinxhref{https://doi.org/10.1021/NL502998Z/SUPPL\{\textbackslash{}\_\}FILE/NL502998Z\{\textbackslash{}\_\}SI\{\textbackslash{}\_\}001.PDF}{doi:10.1021/NL502998Z/SUPPL\{\textbackslash{}\_\}FILE/NL502998Z\{\textbackslash{}\_\}SI\{\textbackslash{}\_\}001.PDF}.
\bibitem[104]{bib:id193}
\sphinxAtStartPar
Yao\sphinxhyphen{}Wei Huang, Ho Wai Howard Lee, Ruzan Sokhoyan, Ragip A. Pala, Krishnan Thyagarajan, Seunghoon Han, Din Ping Tsai, and Harry A. Atwater. Gate\sphinxhyphen{}Tunable Conducting Oxide Metasurfaces. \sphinxstyleemphasis{Nano Letters}, 16(9):5319–5325, 9 2016. \sphinxhref{https://doi.org/10.1021/acs.nanolett.6b00555}{doi:10.1021/acs.nanolett.6b00555}.
\bibitem[105]{bib:id197}
\sphinxAtStartPar
Jinqiannan Zhang, Jingyi Yang, Michael Schell, Aleksei Anopchenko, Long Tao, Zhongyuan Yu, and Ho Wai Howard Lee. Gate\sphinxhyphen{}tunable optical filter based on conducting oxide metasurface heterostructure. \sphinxstyleemphasis{Optics Letters}, 44(15):3653, 8 2019. \sphinxhref{https://doi.org/10.1364/OL.44.003653}{doi:10.1364/OL.44.003653}.
\bibitem[106]{bib:id196}
\sphinxAtStartPar
Hongwei Zhao, Ran Zhang, Hamid T. Chorsi, Wesley A. Britton, Yuyao Chen, Prasad P. Iyer, Jon A. Schuller, Luca Dal Negro, and Jonathan Klamkin. Gate\sphinxhyphen{}tunable metafilm absorber based on indium silicon oxide. \sphinxstyleemphasis{Nanophotonics}, 8(10):1803–1810, 9 2019. \sphinxhref{https://doi.org/10.1515/nanoph-2019-0190}{doi:10.1515/nanoph\sphinxhyphen{}2019\sphinxhyphen{}0190}.
\bibitem[107]{bib:id147}
\sphinxAtStartPar
Junghyun Park, Ju\sphinxhyphen{}Hyung Kang, Soo Jin Kim, Xiaoge Liu, and Mark L. Brongersma. Dynamic Reflection Phase and Polarization Control in Metasurfaces. \sphinxstyleemphasis{Nano Letters}, 17(1):407–413, 1 2017. \sphinxhref{https://doi.org/10.1021/acs.nanolett.6b04378}{doi:10.1021/acs.nanolett.6b04378}.
\bibitem[108]{bib:id145}
\sphinxAtStartPar
Ghazaleh Kafaie Shirmanesh, Ruzan Sokhoyan, Ragip A. Pala, and Harry A. Atwater. Dual\sphinxhyphen{}Gated Active Metasurface at 1550 nm with Wide (300\$\textasciicircum{}\textbackslash{}circ \$) Phase Tunability. \sphinxstyleemphasis{Nano Letters}, 18(5):2957–2963, 5 2018. \sphinxhref{https://doi.org/10.1021/acs.nanolett.8b00351}{doi:10.1021/acs.nanolett.8b00351}.
\bibitem[109]{bib:id465}
\sphinxAtStartPar
Sun Il Kim, Junghyun Park, Byung Gil Jeong, Duhyun Lee, Ki Yeon Yang, Yong Young Park, Kyoungho Ha, and Hyuck Choo. Two\sphinxhyphen{}dimensional beam steering with tunable metasurface in infrared regime. \sphinxstyleemphasis{Nanophotonics}, 11(11):2719–2726, 6 2022. \sphinxhref{https://doi.org/10.1515/NANOPH-2021-0664}{doi:10.1515/NANOPH\sphinxhyphen{}2021\sphinxhyphen{}0664}.
\bibitem[110]{bib:id166}
\sphinxAtStartPar
Ghazaleh Kafaie Shirmanesh, Ruzan Sokhoyan, Pin Chieh Wu, and Harry A. Atwater. Electro\sphinxhyphen{}optically Tunable Multifunctional Metasurfaces. \sphinxstyleemphasis{ACS Nano}, 14(6):6912–6920, 6 2020. \sphinxhref{https://doi.org/10.1021/acsnano.0c01269}{doi:10.1021/acsnano.0c01269}.
\bibitem[111]{bib:id275}
\sphinxAtStartPar
Carlos Errando\sphinxhyphen{}Herranz, Nicolas Le Thomas, and Kristinn B Gylfason. Low\sphinxhyphen{}power optical beam steering by microelectromechanical waveguide gratings. \sphinxstyleemphasis{Optica}, 2019. \sphinxhref{https://doi.org/10.1364/OL.44.000855}{doi:10.1364/OL.44.000855}.
\bibitem[112]{bib:id283}
\sphinxAtStartPar
Yuhua Chang, Jingxuan Wei, and Chengkuo Lee. Metamaterials – from fundamentals and MEMS tuning mechanisms to applications. \sphinxstyleemphasis{Nanophotonics}, 9(10):3049–3070, 8 2020. \sphinxhref{https://doi.org/10.1515/nanoph-2020-0045}{doi:10.1515/nanoph\sphinxhyphen{}2020\sphinxhyphen{}0045}.
\bibitem[113]{bib:id371}
\sphinxAtStartPar
Zang Guanxing, Ziji Liu, Wenjun Deng, and Weiming Zhu. Reconfigurable metasurfaces with mechanical actuations: towards flexible and tunable photonic devices. \sphinxstyleemphasis{Journal of Optics}, 23(1):013001, 1 2021. \sphinxhref{https://doi.org/10.1088/2040-8986/abcc52}{doi:10.1088/2040\sphinxhyphen{}8986/abcc52}.
\bibitem[114]{bib:id68}
\sphinxAtStartPar
Prakash Pitchappa, Manukumara Manjappa, Chong Pei Ho, Ranjan Singh, Navab Singh, and Chengkuo Lee. Active Control of Electromagnetically Induced Transparency Analog in Terahertz MEMS Metamaterial. \sphinxstyleemphasis{Advanced Optical Materials}, 4(4):541–547, 4 2016. \sphinxhref{https://doi.org/10.1002/adom.201500676}{doi:10.1002/adom.201500676}.
\bibitem[115]{bib:id277}
\sphinxAtStartPar
Ehsan Arbabi, Amir Arbabi, Seyedeh Mahsa Kamali, Yu Horie, Mohammad Sadegh Faraji\sphinxhyphen{}Dana, and Andrei Faraon. MEMS\sphinxhyphen{}tunable dielectric metasurface lens. \sphinxstyleemphasis{Nature Communications 2018 9:1}, 9(1):1–9, 2 2018. \sphinxhref{https://doi.org/10.1038/s41467-018-03155-6}{doi:10.1038/s41467\sphinxhyphen{}018\sphinxhyphen{}03155\sphinxhyphen{}6}.
\bibitem[116]{bib:id37}
\sphinxAtStartPar
Youmin Wang, Guangya Zhou, Xiaosheng Zhang, Kyungmok Kwon, Pierre\sphinxhyphen{}A. Blanche, Nicholas Triesault, Kyoung\sphinxhyphen{}sik Yu, and Ming C. Wu. 2D broadband beamsteering with large\sphinxhyphen{}scale MEMS optical phased array. \sphinxstyleemphasis{Optica}, 6(5):557, 5 2019. \sphinxhref{https://doi.org/10.1364/OPTICA.6.000557}{doi:10.1364/OPTICA.6.000557}.
\bibitem[117]{bib:id368}
\sphinxAtStartPar
Shaowei He, Huimin Yang, Yunhui Jiang, Wenjun Deng, and Weiming Zhu. Recent Advances in MEMS Metasurfaces and Their Applications on Tunable Lens. \sphinxstyleemphasis{Micromachines}, 10(8):505, 7 2019. \sphinxhref{https://doi.org/10.3390/mi10080505}{doi:10.3390/mi10080505}.
\bibitem[118]{bib:id146}
\sphinxAtStartPar
Tapashree Roy, Shuyan Zhang, Il Woong Jung, Mariano Troccoli, Federico Capasso, and Daniel Lopez. Dynamic metasurface lens based on MEMS technology. \sphinxstyleemphasis{APL Photonics}, 3(2):21302, 2 2018. \sphinxhref{https://doi.org/10.1063/1.5018865}{doi:10.1063/1.5018865}.
\bibitem[119]{bib:id279}
\sphinxAtStartPar
Christopher A. Dirdal, Paul C. V. Thrane, Firehun T. Dullo, Jo Gjessing, Anand Summanwar, and Jon Tschudi. MEMS\sphinxhyphen{}tunable dielectric metasurface lens using thin\sphinxhyphen{}film PZT for large displacements at low voltages. \sphinxstyleemphasis{Optics Letters}, 47(5):1049, 3 2022. \sphinxhref{https://doi.org/10.1364/OL.451750}{doi:10.1364/OL.451750}.
\bibitem[120]{bib:id418}
\sphinxAtStartPar
Aaron L. Holsteen, Ahmet Fatih Cihan, and Mark L. Brongersma. Temporal color mixing and dynamic beam shaping with silicon metasurfaces. \sphinxstyleemphasis{Science}, 365(6450):257–260, 7 2019. \sphinxhref{https://doi.org/10.1126/science.aax5961}{doi:10.1126/science.aax5961}.
\bibitem[121]{bib:id289}
\sphinxAtStartPar
Prakash Pitchappa, Chong Pei Ho, You Qian, Lokesh Dhakar, Navab Singh, and Chengkuo Lee. Microelectromechanically tunable multiband metamaterial with preserved isotropy. \sphinxstyleemphasis{Scientific Reports}, 5(1):11678, 12 2015. \sphinxhref{https://doi.org/10.1038/srep11678}{doi:10.1038/srep11678}.
\bibitem[122]{bib:id39}
\sphinxAtStartPar
Byung\sphinxhyphen{}Wook Yoo, Mischa Megens, Tianbo Sun, Weijian Yang, Connie J. Chang\sphinxhyphen{}Hasnain, David A. Horsley, and Ming C. Wu. A 32 × 32 optical phased array using polysilicon sub\sphinxhyphen{}wavelength high\sphinxhyphen{}contrast\sphinxhyphen{}grating mirrors. \sphinxstyleemphasis{Optics Express}, 22(16):19029, 8 2014. \sphinxhref{https://doi.org/10.1364/OE.22.019029}{doi:10.1364/OE.22.019029}.
\bibitem[123]{bib:id75}
\sphinxAtStartPar
Longqing Cong, Prakash Pitchappa, Chengkuo Lee, and Ranjan Singh. Active Phase Transition via Loss Engineering in a Terahertz MEMS Metamaterial. \sphinxstyleemphasis{Advanced Materials}, 7 2017. \sphinxhref{https://doi.org/10.1002/adma.201700733}{doi:10.1002/adma.201700733}.
\bibitem[124]{bib:id74}
\sphinxAtStartPar
Longqing Cong, Prakash Pitchappa, Yang Wu, Lin Ke, Chengkuo Lee, Navab Singh, Hyunsoo Yang, and Ranjan Singh. Active Multifunctional Microelectromechanical System Metadevices: Applications in Polarization Control, Wavefront Deflection, and Holograms. \sphinxstyleemphasis{Advanced Optical Materials}, 5(2):1600716, 1 2017. \sphinxhref{https://doi.org/10.1002/adom.201600716}{doi:10.1002/adom.201600716}.
\bibitem[125]{bib:id303}
\sphinxAtStartPar
H. G. Craighead. Nanoelectromechanical systems. \sphinxstyleemphasis{Science}, 290(5496):1532–1535, 11 2000. \sphinxhref{https://doi.org/10.1126/SCIENCE.290.5496.1532}{doi:10.1126/SCIENCE.290.5496.1532}.
\end{sphinxthebibliography}







\renewcommand{\indexname}{Index}
\printindex
\end{document}