%% Generated by Sphinx.
\def\sphinxdocclass{jupyterBook}
\documentclass[a4paper,10pt,english,openany,oneside]{jupyterBook}
\ifdefined\pdfpxdimen
   \let\sphinxpxdimen\pdfpxdimen\else\newdimen\sphinxpxdimen
\fi \sphinxpxdimen=.75bp\relax
\ifdefined\pdfimageresolution
    \pdfimageresolution= \numexpr \dimexpr1in\relax/\sphinxpxdimen\relax
\fi
%% let collapsible pdf bookmarks panel have high depth per default
\PassOptionsToPackage{bookmarksdepth=5}{hyperref}
%% turn off hyperref patch of \index as sphinx.xdy xindy module takes care of
%% suitable \hyperpage mark-up, working around hyperref-xindy incompatibility
\PassOptionsToPackage{hyperindex=false}{hyperref}
%% memoir class requires extra handling
\makeatletter\@ifclassloaded{memoir}
{\ifdefined\memhyperindexfalse\memhyperindexfalse\fi}{}\makeatother

\PassOptionsToPackage{warn}{textcomp}

\catcode`^^^^00a0\active\protected\def^^^^00a0{\leavevmode\nobreak\ }
\usepackage{cmap}
\usepackage{fontspec}
\defaultfontfeatures[\rmfamily,\sffamily,\ttfamily]{}
\usepackage{amsmath,amssymb,amstext}
\usepackage{polyglossia}
\setmainlanguage{english}



\setmainfont{FreeSerif}[
  Extension      = .otf,
  UprightFont    = *,
  ItalicFont     = *Italic,
  BoldFont       = *Bold,
  BoldItalicFont = *BoldItalic
]
\setsansfont{FreeSans}[
  Extension      = .otf,
  UprightFont    = *,
  ItalicFont     = *Oblique,
  BoldFont       = *Bold,
  BoldItalicFont = *BoldOblique,
]
\setmonofont{FreeMono}[
  Extension      = .otf,
  UprightFont    = *,
  ItalicFont     = *Oblique,
  BoldFont       = *Bold,
  BoldItalicFont = *BoldOblique,
]



\usepackage[Bjarne]{fncychap}
\usepackage[,numfigreset=1,mathnumfig]{sphinx}

\fvset{fontsize=\small}
\usepackage{geometry}


% Include hyperref last.
\usepackage{hyperref}
% Fix anchor placement for figures with captions.
\usepackage{hypcap}% it must be loaded after hyperref.
% Set up styles of URL: it should be placed after hyperref.
\urlstyle{same}


\usepackage{sphinxmessages}


\usepackage{etoolbox}
\AtBeginEnvironment{figure}{\pretocmd{\hyperlink}{\protect}{}{}}  

        % Start of preamble defined in sphinx-jupyterbook-latex %
         \usepackage[Latin,Greek]{ucharclasses}
        \usepackage{unicode-math}
        % fixing title of the toc
        \addto\captionsenglish{\renewcommand{\contentsname}{Contents}}
        \hypersetup{
            pdfencoding=auto,
            psdextra
        }
        % End of preamble defined in sphinx-jupyterbook-latex %
        

\title{Video Holography}
\date{Oct 04, 2023}
\release{}
\author{Jan Genoe}
\newcommand{\sphinxlogo}{\sphinxincludegraphics{logo.png}\par}
\renewcommand{\releasename}{}
\makeindex
\begin{document}

\pagestyle{empty}
\sphinxmaketitle
\pagestyle{plain}
\sphinxtableofcontents
\pagestyle{normal}
\phantomsection\label{\detokenize{intro2::doc}}


\begin{DUlineblock}{0em}
\item[] \sphinxstylestrong{\Large Introduction}
\end{DUlineblock}

\sphinxAtStartPar
Today, despite many efforts by researchers world\sphinxhyphen{}wide, there are no holographic projectors that allow video\sphinxhyphen{}rate electronically controlled projection of complex holograms. Optically re\sphinxhyphen{}write\sphinxhyphen{}able holograms exist, but they are too slow; Acoustically\sphinxhyphen{}formed holograms can be switched fast but the image complexity is very limited. We identify the essential roadblock as one that we intend to clear by a breakthrough innovation coming from a combination of electronics, optics and material science.
We propose a radically novel way to make and control holograms, that will be based on the direct, analog, nanometer\sphinxhyphen{}resolution and nanosecond\sphinxhyphen{}speed control over the local refractive index of a slab waveguide core over several square centimetres. Holograms will be formed by leaky waves evanescent from the waveguide, and controlled by the refractive\sphinxhyphen{}index modulation profile in the core. That profile will be controlled and modulated by electrical fields applied with nano\sphinxhyphen{}precision through one of the cladding layers of the waveguide. To that end, a novel metamaterial is proposed for this cladding. Also novel driving schemes will be needed to control the new holographic projecting method.
With this combined radical innovation in architecture, materials and driving schemes, it is the goal of this project to fully prove the concept of video\sphinxhyphen{}rate electrically\sphinxhyphen{}controlled holographic projection. This will be the basis for many future innovations and applications, in domains such as augmented reality, automotive, optical metrology (LIDAR, microscopy, …), mobile communication, education, safety, etc…, and result in a high economic and social impact.

\sphinxAtStartPar
\hyperref[\detokenize{intro2:scalingroadmap}]{Fig.\@ \ref{\detokenize{intro2:scalingroadmap}}} shows the evolution of hologram demonstrators, both in resolution and speed.

\begin{figure}[htbp]
\centering
\capstart

\noindent\sphinxincludegraphics[scale=1.0]{{video-holography}.jpg}
\caption{Scaling Roadmap}\label{\detokenize{intro2:scalingroadmap}}\end{figure}

\sphinxAtStartPar
\hyperref[\detokenize{intro2:selectedimplementation}]{Fig.\@ \ref{\detokenize{intro2:selectedimplementation}}} shows the selected implementation that has been elaborated in this project. A 500 nm thick metamaterial separates the metal electrodes where the requested hologram is applied from the BTO waveguide. As a consequence, no metal is in the presence of the BTO waveguide, which allows light to propagate in the waveguide without scattering losses. The metamaterial is fabricated using SiN wherein vertical trenches are etched at 100 nm pitch in both directions. These trenches have been filled with InGaZnO that has been engineered such that the dielectric constant carefully matches the dielectric constant of SiN.  This lead to a metamaterial that is completely uniform and has no losses, when it is considered as the optical material that forms the cladding of the waveguide. However, when the same metamaterial is looked at from the electrical perspective, we have conductive channels at a 100 nm pitch that guides the electrical signal from the electrical contacts below to the waveguide above.
As the waveguide material, BTO has been selected. BTO is known to have the highest Pockels effect. This enables us to alter the effective dielectric constant of the waveguide locally at a pitch of 100 nm using relative small electric fields. This dielectric variation forms the hologram that is applied.

\begin{figure}[htbp]
\centering
\capstart

\noindent\sphinxincludegraphics[scale=1.0]{{video-holography2}.jpg}
\caption{Selected implementation}\label{\detokenize{intro2:selectedimplementation}}\end{figure}

\sphinxAtStartPar
Changing the hologram using the hardware in \hyperref[\detokenize{intro2:selectedimplementation}]{Fig.\@ \ref{\detokenize{intro2:selectedimplementation}}} can be obtained by changing the voltages at the bottom electrodes, which can be done reasonably fast, e.g. at 100 Hz rate. This allows also to swap the hologram between 3 subsequent holograms, one for red, one for green and one for blue at video rates. This yields full color video holography.

\begin{DUlineblock}{0em}
\item[] \sphinxstylestrong{\large Main project results}
\end{DUlineblock}

\begin{DUlineblock}{0em}
\item[] \sphinxstylestrong{\large Metamaterial development}
\end{DUlineblock}

\begin{DUlineblock}{0em}
\item[] \sphinxstylestrong{\large High\sphinxhyphen{}quality BTO waveguide}
\end{DUlineblock}

\begin{DUlineblock}{0em}
\item[] \sphinxstylestrong{\large Remaining challenges en further work}
\end{DUlineblock}

\sphinxAtStartPar
The control of the BTO waveguide at 100 nm resolution requires close interaction with the metamaterial. Our simulations (see {[}\hyperlink{cite.bib:id4}{1}{]}) indicate that when the separation between the BTO and the metamaterial goes beyond 5 nm, the effective control is too low for an efficient demonstrator. Therefor, we targeted an oxide\sphinxhyphen{}oxide bonding process yielding an separation below 2 nm. Although other demonstrators of oxide\sphinxhyphen{}oxide bonding, also in our lab, have indicated that this should be in reach, the practical between the BTO wafer and the metamaterial wafer has not yet been possible.

\begin{DUlineblock}{0em}
\item[] \sphinxstylestrong{\large Main funding info}
\end{DUlineblock}
\begin{itemize}
\item {} 
\sphinxAtStartPar
Programme Funding: Horizon 2020

\item {} 
\sphinxAtStartPar
Sub Programme Area: ERC\sphinxhyphen{}2016\sphinxhyphen{}ADG

\item {} 
\sphinxAtStartPar
Project Reference: 742299

\item {} 
\sphinxAtStartPar
From October 1, 2017 to March 31, 2023

\item {} 
\sphinxAtStartPar
Budget: EUR 2 499 074

\item {} 
\sphinxAtStartPar
Contract type: ERC\sphinxhyphen{}ADG

\end{itemize}

\sphinxstepscope


\chapter{Team}
\label{\detokenize{Team:team}}\label{\detokenize{Team::doc}}

\section{Core Team}
\label{\detokenize{Team:core-team}}

\subsection{Principal Investigator}
\label{\detokenize{Team:principal-investigator}}

\begin{savenotes}\sphinxattablestart
\centering
\begin{tabular}[t]{|\X{25}{125}|\X{100}{125}|}
\hline

\sphinxAtStartPar
\sphinxincludegraphics{{JanGenoe}.jpg}
&
\sphinxAtStartPar
Prof. Jan Genoe is scientific director at the Host institution imec and has received all support from the Host institution to build the research team and execute the research. Prof. Jan Genoe also takes the scientific leadership of the Video Holography ERC research.
\\
\hline
\end{tabular}
\par
\sphinxattableend\end{savenotes}


\subsection{Senior academic staff in the team}
\label{\detokenize{Team:senior-academic-staff-in-the-team}}

\begin{savenotes}\sphinxattablestart
\centering
\sphinxcapstartof{table}
\sphinxthecaptionisattop
\sphinxcaption{Senior academic staff}\label{\detokenize{Team:id1}}
\sphinxaftertopcaption
\begin{tabular}[t]{|\X{25}{125}|\X{100}{125}|}
\hline

\sphinxAtStartPar
\sphinxincludegraphics{{Robert}.jpg}
&
\sphinxAtStartPar
Dr. Robert Gehlhaar provides scientific input on the optical stack design and characterization.
\\
\hline
\sphinxAtStartPar
\sphinxincludegraphics{{Zsolt}.jpg}
&
\sphinxAtStartPar
Dr. Zsolt Tokei provides technology input on the realisation of devices in the 300mm cleanroom.
\\
\hline
\sphinxAtStartPar
\sphinxincludegraphics{{Clement}.jpg}
&
\sphinxAtStartPar
Prof. Clement Merckling provides scientific input on the material growth conditions for the BTO and STO waveguide materials.
\\
\hline
\sphinxAtStartPar
\sphinxincludegraphics{{Paul}.jpg}
&
\sphinxAtStartPar
Prof. Paul Heremans provides scientific input on the device performance.
\\
\hline
\end{tabular}
\par
\sphinxattableend\end{savenotes}


\subsection{PhD students}
\label{\detokenize{Team:phd-students}}

\begin{savenotes}\sphinxattablestart
\centering
\sphinxcapstartof{table}
\sphinxthecaptionisattop
\sphinxcaption{PhD students}\label{\detokenize{Team:id2}}
\sphinxaftertopcaption
\begin{tabular}[t]{|\X{25}{125}|\X{100}{125}|}
\hline

\sphinxAtStartPar
\sphinxincludegraphics{{Guillaume}.jpg}
&
\sphinxAtStartPar
Guillaume Croes is the PhD student elaborating the metamaterial stack and optical model for the optimization for driving the hologram.
\\
\hline
\sphinxAtStartPar
\sphinxincludegraphics{{Tsang-HsuanWang}.jpg}
&
\sphinxAtStartPar
Tsang\sphinxhyphen{}Hsuan Wang is the PhD student elaborating the optimized growth conditions for the BTO and STO waveguide materials.
\\
\hline
\end{tabular}
\par
\sphinxattableend\end{savenotes}


\section{Other contributors}
\label{\detokenize{Team:other-contributors}}\begin{itemize}
\item {} 
\sphinxAtStartPar
Diana Tsvetanova provides input on the CMP processes in the 300mm line.

\item {} 
\sphinxAtStartPar
Yunlong Li provides input on the process sequence in the 300 mm line.

\item {} 
\sphinxAtStartPar
Renauld Puybaret is in charge of the daily supervision of the process in the 300 mm line.

\item {} 
\sphinxAtStartPar
Thomas Raes is in charge of the Mask preparation for the process in the 300 mm line.

\item {} 
\sphinxAtStartPar
Deniz Sabuncuoglu Tezcan is in charge of the supervision of the process in the 300 mm line.

\item {} 
\sphinxAtStartPar
Jeremy Segers is in charge of the oxide\sphinxhyphen{}oxide bonding process between the BTO wafer and the optical transparent metamaterial.

\end{itemize}

\sphinxstepscope


\chapter{Publications}
\label{\detokenize{Publications:publications}}\label{\detokenize{Publications::doc}}

\section{Journal papers}
\label{\detokenize{Publications:journal-papers}}

\begin{savenotes}\sphinxattablestart
\centering
\sphinxcapstartof{table}
\sphinxthecaptionisattop
\sphinxcaption{Journal papers}\label{\detokenize{Publications:id1}}
\sphinxaftertopcaption
\begin{tabular}[t]{|\X{25}{125}|\X{100}{125}|}
\hline

\sphinxAtStartPar
\sphinxincludegraphics{{logo}.png}
&
\sphinxAtStartPar
Tsang\sphinxhyphen{}Hsuan Wang\sphinxhref{http://orcid.org/0000-0002-7760-7500}{},
Po\sphinxhyphen{}Chun Hsu\sphinxhref{http://orcid.org/0000-0003-0823-6088}{},
Maxim Korytov,
Jan Genoe\sphinxhref{http://orcid.org/0000-0002-4019-5979}{},
Clement Merckling\sphinxhref{http://orcid.org/0000-0003-3084-2543}{},
\sphinxstylestrong{\DUrole{xref,myst}{Polarization control of epitaxial barium titanate (BaTiO3) grown by pulsed\sphinxhyphen{}laser deposition on a MBE\sphinxhyphen{}SrTiO3/Si(001) pseudo\sphinxhyphen{}substrate}},
Journal of Applied Physics 128, 104104 (September 2020),
\sphinxhref{http://dx.doi.org/10.1063/5.0019980}{DOI: 10.1063/5.0019980}
\\
\hline
\sphinxAtStartPar
\sphinxincludegraphics{{logo}.png}
&
\sphinxAtStartPar
Tsang\sphinxhyphen{}Hsuan Wang\sphinxhref{http://orcid.org/0000-0002-7760-7500}{},
Robert Gehlhaar\sphinxhref{http://orcid.org/0000-0002-3038-9462}{},
Thierry Conard,
Paola Favia,
Jan Genoe\sphinxhref{http://orcid.org/0000-0002-4019-5979}{},
Clement Merckling\sphinxhref{http://orcid.org/0000-0003-3084-2543}{},
\sphinxstylestrong{\DUrole{xref,myst}{Interfacial control of SrTiO3/Si(0 0 1) epitaxy and its effect on physical and optical properties}},
Journal of Crystal Growth 582, 126524 (March 2022),
\sphinxhref{http://dx.doi.org/10.1016/j.jcrysgro.2022.126524}{DOI: 10.1016/j.jcrysgro.2022.126524}
\\
\hline
\sphinxAtStartPar
\sphinxincludegraphics{{AO2023}.jpg}
&
\sphinxAtStartPar
Guillaume Croes\sphinxhref{http://orcid.org/0000-0001-6168-9794}{},
Renaud Puybaret\sphinxhref{http://orcid.org/0000-0002-4946-2658}{},
Janusz Bogdanowicz,
Umberto Celano,
Robert Gehlhaar\sphinxhref{http://orcid.org/0000-0002-3038-9462}{},
Jan Genoe\sphinxhref{http://orcid.org/0000-0002-4019-5979}{},
\sphinxstylestrong{\DUrole{xref,download,myst}{Photonic Metamaterial with a Subwavelength Electrode Pattern}},
Applied Optics 62,F14 (March 2023),
\sphinxhref{http://dx.doi.org/10.1364/AO.481396}{DOI: 10.1364/AO.481396}
\\
\hline
\end{tabular}
\par
\sphinxattableend\end{savenotes}


\section{Conferences}
\label{\detokenize{Publications:conferences}}

\begin{savenotes}\sphinxattablestart
\centering
\sphinxcapstartof{table}
\sphinxthecaptionisattop
\sphinxcaption{Conferences}\label{\detokenize{Publications:id2}}
\sphinxaftertopcaption
\begin{tabular}[t]{|\X{25}{125}|\X{100}{125}|}
\hline

\sphinxAtStartPar
\sphinxincludegraphics{{logo}.png}
&
\sphinxAtStartPar
Artur Hermans,
Robby Janneck,
Cedric Rolin\sphinxhref{http://orcid.org/0000-0001-5542-8504}{},
S. Clemmen,
Paul Heremans\sphinxhref{http://orcid.org/0000-0003-2151-1718}{},
Jan Genoe\sphinxhref{http://orcid.org/0000-0002-4019-5979}{},
Roel Baets,
\sphinxstylestrong{Growth of Thin Film Organic Crystals with Strong Nonlinearity for On\sphinxhyphen{}Chip Second\sphinxhyphen{}Order Nonlinear Optics},
Proc. IEEE Photonics Benelux Symposium, Brussels, Belgium, November 15\sphinxhyphen{}16, 2018.
\\
\hline
\sphinxAtStartPar
\sphinxincludegraphics{{logo}.png}
&
\sphinxAtStartPar
Guillaume Croes\sphinxhref{http://orcid.org/0000-0001-6168-9794}{},
Nikolay Smolentsev,
Tsang\sphinxhyphen{}Hsuan Wang\sphinxhref{http://orcid.org/0000-0002-7760-7500}{},
Robert Gehlhaar\sphinxhref{http://orcid.org/0000-0002-3038-9462}{},
Jan Genoe\sphinxhref{http://orcid.org/0000-0002-4019-5979}{},
\sphinxstylestrong{\DUrole{xref,myst}{Non\sphinxhyphen{}linear electro\sphinxhyphen{}optic modelling of a Barium Titanate grating coupler}},
Proc. SPIE 11484, 114840D: Optical Modeling and Performance Predictions XI (August 2020),
\sphinxhref{http://dx.doi.org/10.1117/12.2568032}{DOI: 10.1117/12.2568032}
\\
\hline
\sphinxAtStartPar
\sphinxincludegraphics{{logo}.png}
&
\sphinxAtStartPar
Guillaume Croes\sphinxhref{http://orcid.org/0000-0001-6168-9794}{},
Robert Gehlhaar\sphinxhref{http://orcid.org/0000-0002-3038-9462}{},
Jan Genoe\sphinxhref{http://orcid.org/0000-0002-4019-5979}{},
\sphinxstylestrong{Hologram Wavefront Shaping by a Non\sphinxhyphen{}Linear Electro\sphinxhyphen{}Optic Spatial Light Modulator},
Holography: Advances and Modern Trends VIII, April 2023, Prague, Czech Republic
\\
\hline
\sphinxAtStartPar
\sphinxincludegraphics{{logo}.png}
&
\sphinxAtStartPar
Guillaume Croes\sphinxhref{http://orcid.org/0000-0001-6168-9794}{},
Robert Gehlhaar\sphinxhref{http://orcid.org/0000-0002-3038-9462}{},
Jan Genoe\sphinxhref{http://orcid.org/0000-0002-4019-5979}{},
\sphinxstylestrong{Sub\sphinxhyphen{}Wavelength Custom Reprogrammable Active Photonic Platform for High\sphinxhyphen{}Resolution Beam Shaping and Holography},
Proc. SPIE PC12196, PC1219619: Active Photonic Platforms, San Diego, California, United States (October 2022)
\\
\hline
\sphinxAtStartPar
\sphinxincludegraphics{{logo}.png}
&
\sphinxAtStartPar
Clement Merckling\sphinxhref{http://orcid.org/0000-0003-3084-2543}{},
Islam Ahmed,
Tsang\sphinxhyphen{}Hsuan Wang\sphinxhref{http://orcid.org/0000-0002-7760-7500}{},
Moloud Kaviani,
Jan Genoe\sphinxhref{http://orcid.org/0000-0002-4019-5979}{},
Stefan De Gendt\sphinxhref{http://orcid.org/0000-0003-3775-3578}{},
\sphinxstylestrong{Integrated Perovskites Oxides on Silicon: From Optical to Quantum Applications},
ECS Meeting Abstracts MA2022\sphinxhyphen{}01, 1060 , July 2022,
\sphinxhref{http://dx.doi.org/10.1149/MA2022-01191060mtgabs}{DOI: 10.1149/MA2022\sphinxhyphen{}01191060mtgabs}
\\
\hline
\sphinxAtStartPar
\sphinxincludegraphics{{logo}.png}
&
\sphinxAtStartPar
Tsang\sphinxhyphen{}Hsuan Wang\sphinxhref{http://orcid.org/0000-0002-7760-7500}{},
Robert Gehlhaar\sphinxhref{http://orcid.org/0000-0002-3038-9462}{},
Thierry Conard,
Jan Genoe\sphinxhref{http://orcid.org/0000-0002-4019-5979}{},
Clement Merckling\sphinxhref{http://orcid.org/0000-0003-3084-2543}{},
\sphinxstylestrong{Interface Control and Characterization of SrTiO3/Si(001)},
Proc. E\sphinxhyphen{}MRS\sphinxhyphen{}fall, 20th to 23rd September 2021
\\
\hline
\end{tabular}
\par
\sphinxattableend\end{savenotes}


\section{PhD thesis}
\label{\detokenize{Publications:phd-thesis}}

\begin{savenotes}\sphinxattablestart
\centering
\sphinxcapstartof{table}
\sphinxthecaptionisattop
\sphinxcaption{PhD thesis}\label{\detokenize{Publications:id3}}
\sphinxaftertopcaption
\begin{tabular}[t]{|\X{25}{125}|\X{100}{125}|}
\hline

\sphinxAtStartPar
\sphinxincludegraphics{{phd}.png}
&
\sphinxAtStartPar
Tsang\sphinxhyphen{}Hsuan Wang,
\sphinxstylestrong{\DUrole{xref,download,myst}{Study of Barium Titanate Epitaxy on Silicon toward Its Application in Video Holography}},
PhD Thesis, KULeuven, Leuven, Belgium, Monday, February 13, 2023.
\\
\hline
\sphinxAtStartPar
\sphinxincludegraphics{{phd}.png}
&
\sphinxAtStartPar
Guillaume Croes, (PhD Thesis in preparation), KULeuven, Leuven, Belgium
\\
\hline
\end{tabular}
\par
\sphinxattableend\end{savenotes}

\sphinxstepscope

\begin{sphinxthebibliography}{1}
\bibitem[1]{bib:id4}
\sphinxAtStartPar
Guillaume Croes, Robert Gehlhaar, and Jan Genoe. Sub\sphinxhyphen{}wavelength custom reprogrammable active photonic platform for high\sphinxhyphen{}resolution beam shaping and holography. In \sphinxstyleemphasis{Active Photonic Platforms 2022}, volume PC12196, PC1219619. San Diego, California, United States, October 2022. SPIE. \sphinxhref{https://doi.org/10.1117/12.2632022}{doi:10.1117/12.2632022}.
\end{sphinxthebibliography}







\renewcommand{\indexname}{Index}
\printindex
\end{document}