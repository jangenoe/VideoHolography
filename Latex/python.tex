%% Generated by Sphinx.
\def\sphinxdocclass{jupyterBook}
\documentclass[a4paper,10pt,english,openany,oneside]{jupyterBook}
\ifdefined\pdfpxdimen
   \let\sphinxpxdimen\pdfpxdimen\else\newdimen\sphinxpxdimen
\fi \sphinxpxdimen=.75bp\relax
\ifdefined\pdfimageresolution
    \pdfimageresolution= \numexpr \dimexpr1in\relax/\sphinxpxdimen\relax
\fi
%% let collapsible pdf bookmarks panel have high depth per default
\PassOptionsToPackage{bookmarksdepth=5}{hyperref}
%% turn off hyperref patch of \index as sphinx.xdy xindy module takes care of
%% suitable \hyperpage mark-up, working around hyperref-xindy incompatibility
\PassOptionsToPackage{hyperindex=false}{hyperref}
%% memoir class requires extra handling
\makeatletter\@ifclassloaded{memoir}
{\ifdefined\memhyperindexfalse\memhyperindexfalse\fi}{}\makeatother

\PassOptionsToPackage{warn}{textcomp}

\catcode`^^^^00a0\active\protected\def^^^^00a0{\leavevmode\nobreak\ }
\usepackage{cmap}
\usepackage{fontspec}
\defaultfontfeatures[\rmfamily,\sffamily,\ttfamily]{}
\usepackage{amsmath,amssymb,amstext}
\usepackage{polyglossia}
\setmainlanguage{english}



\setmainfont{FreeSerif}[
  Extension      = .otf,
  UprightFont    = *,
  ItalicFont     = *Italic,
  BoldFont       = *Bold,
  BoldItalicFont = *BoldItalic
]
\setsansfont{FreeSans}[
  Extension      = .otf,
  UprightFont    = *,
  ItalicFont     = *Oblique,
  BoldFont       = *Bold,
  BoldItalicFont = *BoldOblique,
]
\setmonofont{FreeMono}[
  Extension      = .otf,
  UprightFont    = *,
  ItalicFont     = *Oblique,
  BoldFont       = *Bold,
  BoldItalicFont = *BoldOblique,
]



\usepackage[Bjarne]{fncychap}
\usepackage[,numfigreset=1,mathnumfig]{sphinx}

\fvset{fontsize=\small}
\usepackage{geometry}


% Include hyperref last.
\usepackage{hyperref}
% Fix anchor placement for figures with captions.
\usepackage{hypcap}% it must be loaded after hyperref.
% Set up styles of URL: it should be placed after hyperref.
\urlstyle{same}


\usepackage{sphinxmessages}


\usepackage{etoolbox}
\AtBeginEnvironment{figure}{\pretocmd{\hyperlink}{\protect}{}{}}  

        % Start of preamble defined in sphinx-jupyterbook-latex %
         \usepackage[Latin,Greek]{ucharclasses}
        \usepackage{unicode-math}
        % fixing title of the toc
        \addto\captionsenglish{\renewcommand{\contentsname}{Contents}}
        \hypersetup{
            pdfencoding=auto,
            psdextra
        }
        % End of preamble defined in sphinx-jupyterbook-latex %
        

\title{Video Holography}
\date{Oct 02, 2023}
\release{}
\author{Jan Genoe}
\newcommand{\sphinxlogo}{\sphinxincludegraphics{logo.png}\par}
\renewcommand{\releasename}{}
\makeindex
\begin{document}

\pagestyle{empty}
\sphinxmaketitle
\pagestyle{plain}
\sphinxtableofcontents
\pagestyle{normal}
\phantomsection\label{\detokenize{intro::doc}}


\begin{DUlineblock}{0em}
\item[] \sphinxstylestrong{\Large Introduction}
\end{DUlineblock}

\sphinxAtStartPar
Today, despite many efforts by researchers world\sphinxhyphen{}wide, there are no holographic projectors that allow video\sphinxhyphen{}rate electronically controlled projection of complex holograms. Optically re\sphinxhyphen{}write\sphinxhyphen{}able holograms exist, but they are too slow; Acoustically\sphinxhyphen{}formed holograms can be switched fast but the image complexity is very limited. We identify the essential roadblock as one that we intend to clear by a breakthrough innovation coming from a combination of electronics, optics and material science.
We propose a radically novel way to make and control holograms, that will be based on the direct, analog, nanometer\sphinxhyphen{}resolution and nanosecond\sphinxhyphen{}speed control over the local refractive index of a slab waveguide core over several square centimetres. Holograms will be formed by leaky waves evanescent from the waveguide, and controlled by the refractive\sphinxhyphen{}index modulation profile in the core. That profile will be controlled and modulated by electrical fields applied with nano\sphinxhyphen{}precision through one of the cladding layers of the waveguide. To that end, a novel metamaterial is proposed for this cladding. Also novel driving schemes will be needed to control the new holographic projecting method.
With this combined radical innovation in architecture, materials and driving schemes, it is the goal of this project to fully prove the concept of video\sphinxhyphen{}rate electrically\sphinxhyphen{}controlled holographic projection. This will be the basis for many future innovations and applications, in domains such as augmented reality, automotive, optical metrology (LIDAR, microscopy, …), mobile communication, education, safety, etc…, and result in a high economic and social impact.

\begin{DUlineblock}{0em}
\item[] \sphinxstylestrong{\large project results}
\end{DUlineblock}

\begin{DUlineblock}{0em}
\item[] \sphinxstylestrong{\large Main funding info}
\end{DUlineblock}
\begin{itemize}
\item {} 
\sphinxAtStartPar
Programme Funding: Horizon 2020

\item {} 
\sphinxAtStartPar
Sub Programme Area: ERC\sphinxhyphen{}2016\sphinxhyphen{}ADG

\item {} 
\sphinxAtStartPar
Project Reference: 742299

\item {} 
\sphinxAtStartPar
From 01.10.2017 to 31.03.2023

\item {} 
\sphinxAtStartPar
Budget: EUR 2 499 074

\item {} 
\sphinxAtStartPar
Contract type: ERC\sphinxhyphen{}ADG

\end{itemize}

\sphinxstepscope


\chapter{Video Holography}
\label{\detokenize{Overview:video-holography}}\label{\detokenize{Overview::doc}}

\section{Introduction}
\label{\detokenize{Overview:introduction}}

\section{project results}
\label{\detokenize{Overview:project-results}}
\sphinxstepscope


\chapter{Team}
\label{\detokenize{Team:team}}\label{\detokenize{Team::doc}}

\section{Core Team}
\label{\detokenize{Team:core-team}}

\subsection{Principal Investigator: Jan Genoe}
\label{\detokenize{Team:principal-investigator-jan-genoe}}

\subsection{Senior Staff}
\label{\detokenize{Team:senior-staff}}

\subsection{PhD students}
\label{\detokenize{Team:phd-students}}

\section{Other contributors}
\label{\detokenize{Team:other-contributors}}
\sphinxstepscope


\chapter{Publications}
\label{\detokenize{Publications:publications}}\label{\detokenize{Publications::doc}}

\section{Journal papers}
\label{\detokenize{Publications:journal-papers}}

\section{Conferences}
\label{\detokenize{Publications:conferences}}

\section{PhD thesis}
\label{\detokenize{Publications:phd-thesis}}
\sphinxstepscope
\phantomsection\label{\detokenize{bib:id1}}






\renewcommand{\indexname}{Index}
\printindex
\end{document}